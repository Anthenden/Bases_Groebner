\subsection{G�n�ralit�s} 
\begin{Def}

 Un graphe non orient� est un couple $(S,A)$, o� $S$ est un ensemble fini non vide (des �l�ments sont les sommets ) et $A$ est une partie de l'ensemble $\mathcal{P}_2 (S) $ des paires d'�l�ments de $S $ (les �l�ments de $A $ sont les ar�tes). 
\end{Def}  
\begin{Def} 
Soit $G:=(A,S)$ un graphe non orient�. Les sommets $s,t$ sont dits adjacents si $(s,t) \in A$ 
\end{Def}  
\begin{Def} 
Soit $p\in \mathbb{N}^*$.\\ Notons $C_p=\{x_1,\ldots,x_p\}$ un ensemble de couleurs. Un graphe $G:=(A,S)$ est coloriable si on peut associer � chaque sommet de $G$ une couleur de $C_p$ tel que deux sommets adjacents n'aient pas la m�me couleur.  
\end{Def}
  \subsection{Equations polynomiales} 
	Soit $G:=(A,S)$ un graphe non orient� et $p\in \mathbb{N}^*$. \\ 
	Soit $n:=Card(A)$.\\ 
	Associons � chaque sommet de $G$ la variable $x_i$ et � chaque couleur une racine $p$�me de l'unit� \emph{i.e.} $\forall i \in [\![1,n]\!],x_i^p=1$. \\ On impose de plus, que si $x_i$ et $x_j$ sont adjacents alors $x_i \neq x_j$. Cela revient � dire que $\sum_{k=0}^{p-1} x_i^k x_j^{p-1-k}=0$. \\ En effet, $0=x_i^p-x_j^p=\underbrace{(x_i-x_j)}_{\neq 0}\sum_{k=0}^{p-1} x_i^k x_j^{p-1-k}$.  $G$ est coloriable avec $p$ couleurs si, et seulement si, \\ le syst�me $\begin{cases} \forall i \in [\![1,n]\!],x_i^p=1 \\ \forall i,j \in [\![1,n]\!], x_i \text{ et } x_j \text{ sont adjacents }, \sum_{k=0}^{p-1} x_i^k x_j^{p-1-k}=0 \end{cases}$ a une solution 