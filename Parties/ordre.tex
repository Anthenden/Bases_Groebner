Soit un ensemble $A$ et une relation d'ordre $\leq$ sur $A$.\\

\begin{Def}
On dit que $\leq $ est un bon ordre si toute partie non vide de $A$ admet un plus petit �l�ment, c'est-�-dire :$\forall  C \subset A,C \neq \emptyset, \exists c \in C, \forall b \in B, c \leq b$.
\end{Def}

\begin{Def}

On dit que $\leq $ est un ordre bien fond� si toute partie non vide de $A$ admet un �l�ment minimal, c'est-�-dire :
$\forall  C \subset A,C \neq \emptyset, \exists c \in C, \forall b \in B, b \leq c \Rightarrow c=b$.
\end{Def}

\begin{prop}

Soit $A$ un ensemble et $\leq $ une relation d'ordre sur $A$.

$\leq$ est total et bien fond� ssi $\leq$ est un bon ordre.
\end{prop}
 \begin{proof}
 Supposons que $\leq$ est total et bien fond�.\\
Soit $C \subset A$ non vide.     \\
Alors il existe un �l�ment minimal $c$ de $C$ (bien fond�)  tel que $ \forall b \in B, b \leq c \Rightarrow c=b$.\\
D'o�  $ \forall b \in B, b > c \text{ ou }  c=b$ car $\leq $ est total. \\
c'est-�-dire $ \forall b \in B, b \geq c$.\\
ou encore que $c$ est le plus petit �l�ment que $C$.\\
 $\leq $ est donc un bon ordre.\\
Supposons que $\leq $ est un bon ordre.\\
Soit $x,y \in A$. Alors $\{x,y\}$ admet un plus petit �l�ment et donc $x \leq y$ ou $y \leq x$.\\
$\leq$ est donc total.\\
Soit $C \subset A$ alors il existe $c \in C$ tel que $\forall b \in C, c\leq b$.\\
Alors si $b \leq c$ alors, par antisym�trie, $b=c$.\\
Cela permet d'en d�duire que $\leq $ est un ordre bien fond�.\\
\end{proof}


\begin{prop}


Soit $A$ un ensemble et $\leq $ une relation d'ordre sur $A$.

$\leq$ est bien fond� ssi il n'existe pas de suite infinie strictement d�croissante.
\end{prop}

\begin{proof}
Montrons cet �nonc� par contrapos�e :\\
$\leq $ n'est pas bien fond�e ssi il existe une suite infinie strictement croissante c'est-�-dire
il existe une partie $S$ de $A$ tel que pour tout $c \in S$, il existe $b \in S$ tel que $c > b$ \\
(car $non(A \Rightarrow B) \Leftrightarrow (A \text{ et } non(B))$ et donc $ (b \leq c \Rightarrow c=b) \Leftrightarrow (b \leq c \text{ et } b \neq c)  \Leftrightarrow (b < c)$\\
Soit $\alpha_1 \in S$ alors il existe $\alpha_2 \in S$  tel que $\alpha_1>\alpha_2$.\\
En it�rant ce processus, on construit une suite $(\alpha_i)_{i \in \mathbb{N}}$ strictement d�croissante.\\
R�ciproquement, supposons l'existence d'une telle suite alors l'ensemble $\{a_i | i \in \mathbb{N}\}\subset A$ n'admet pas d'�l�ment minimal donc $\leq$ n'est pas bien fond�.
\end{proof}




