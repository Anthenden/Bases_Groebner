Soient $R$ un anneau commutatif int\`egre de corps de fractions $L$ ainsi que : \\
$A:=\sum_{k=0}^p a_kX^k \in R_p[X]$ et $B:=\sum_{k=0}^q b_kX^k \in R_p[X]$.\\
On appelle la matrice de Sylvester la matrice : \\
$S_{p,q}(A,B)=\begin{pmatrix}
a_0 & 0 &\cdots & 0 & b_0 & 0 &0 & \cdots & 0 & 0 \\
a_1 & a_0& \ddots & 0 &b_1 & b_0 &0 & \ddots & 0 & 0 \\
\vdots & a_1 &\ddots & 0 &\vdots & b_1 &b_0 & \ddots & 0 & 0 \\
\vdots & \vdots &\ddots & a_0 &\vdots & \vdots & b_1 & \ddots & 0 & 0 \\
a_ {p-1} & \vdots& \ddots & a_1 &b_q & \vdots &\vdots & \ddots & b_0 & 0 \\
a_p & a_{p-1} &\ddots & \vdots &0 & b_q &\vdots & \ddots & b_1 & b_0 \\
0 & a_p &\ddots & \vdots &0 & 0 &b_q & \ddots & \vdots & b_1 \\
\vdots & \vdots &\ddots & \vdots &\vdots & \vdots &0 & \ddots & \vdots & \vdots \\
\vdots & \vdots &\ddots & a_{p-1} &\vdots & \vdots &0 & \ddots & b_q & \vdots \\
0 & 0& \ldots & a_p &0 & 0 &0 & \cdots & 0 & b_q \\
\end{pmatrix}$. \\
On notera par $Res_{p,q}(A,B)$ le d\'eterminant de $S_{p,q}(A,B)$
\begin{prop}
$Res_{p,q}(A,B)$ est nul si, et seulement si, il existe $P \in R_{q-1}[X]$ et $Q \in R_{p-1}[X]$ non tous deux nuls tels que $AP+BQ=0$
\end{prop}
\begin{prop}
Il existe $P \in R_{q-1}[X]$ et $Q \in R_{p-1}[X]$ non tous deux nuls tels que $AP+BQ=Res_{p,q}(A,B)$
\end{prop}
\begin{Def}
Le r\'esultant des polyn\^omes $A,B \in R[X]$ de degr\'es respectifs $p,q \geq 0$ est l'\'el\'ement $Res(A,B)=Res_{p,q}(A,B)$
\end{Def}
\begin{Rq}
Lien entre les valeurs de $Res_{p,q}(A,B)$ et de $Res(A,B)$ : 
\begin{itemize}
    \item Si $p=deg(A)$ et $q=deg(B)$ alors, par d\'efinition, $Res(A,B)=Res_{p,q}(A,B)$
    \item Si $p=deg(A)$ et $q>deg(B)$ alors $Res_{p,q}(A,B)=((-1)^p a_p)^{q-deg B}Res(A,B)$
    \item Si $p>deg(A)$ et $q=deg(B)$ alors $Res_{p,q}(A,B)=b_q^{p-deg A}Res(A,B)$
    \item Si $p>deg(A)$ et $q=deg(B)$ alors $Res_{p,q}(A,B)=0$
\end{itemize}
\end{Rq}
\begin{prop}
Soit $A=QB+A_1$ une division euclidienne, avec $A_1 \neq 0$. Alors, avec les m\^emes notations que pr\'ec\'edemment, $Res(A,B)=b_q^{deg(A)-deg(A_1} Res(A_1,B)$
\end{prop}
\begin{Lemme}
Si $B=(X-\beta)*C$, alors $Res(A,B)=A(\beta)Res(A,C)$
\end{Lemme}
\begin{Thm}
Si $A:=a(X-\alpha_1)\ldots (X-\alpha_p)$ et $B:=b(X-\beta_1)\ldots(X-\beta_q)$, alors :
$Res(A,B)=b^pA(\beta_1)\ldots A(\beta_q)$
$=b^pa^q \prod_{i=1}^p \prod_{j=1}^q (\beta_j-\alpha_i)$
$=(-1)^{pq} a^q B(\alpha_1) \ldots B(\alpha_q)$
\end{Thm}
\begin{cor}
Supposons le corps $L$ alg\'ebriquement clos. Alors $Res(A,B)=0$ si, et seulement si, les polyn\^omes $A$ et $B$ ont une racine commune.
\end{cor}

