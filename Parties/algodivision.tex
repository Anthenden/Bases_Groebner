  \begin{Lemme}Soit $\alpha,\alpha_1,\ldots,\alpha_n \in \mathbb{N}^n $ tel que : $X^\alpha >X^{\alpha_1}>\ldots >X^{\alpha_n}$. Soit $f,g \in k[X_1,\ldots,X_n]$ tels que $LT(f)=LT(g)$ alors $LM(f-g) <LT(f)=LT(g)$
	\end{Lemme}
	\begin{demo}
	Soit $f:=pX^\alpha+\sum p_{\alpha_i}X^{\alpha_i}$ et $g:=pX^\alpha+\sum q_{\alpha_i}X^{\alpha_i}$ alors $LM(f-g)=LM(\sum p_{\alpha_i}X^{\alpha_i}) \leq X^{\alpha_1} < X^\alpha=LM(f)=LM(g)$
\end{demo}
%****************************************************************
\begin{Thm}
\begin{algorithm}
\caption{Algorithme de division}
\begin{algorithmic}
\REQUIRE $f_1,\ldots,f_s,f$
\ENSURE $a_1,\ldots,a_s,r$
\STATE $a_1 :=0 ;\ldots ; a_s:=0 ; r:=0$
\STATE $p:=0$
\WHILE{$p \neq 0$}
\STATE $i:=1$
\STATE $divisionoccured:=false $
\WHILE{$i \leq s $ et $divisionoccured = false $}
\IF{$LT (f_i) | LT (p)$}
\STATE $a_i:= a_i+LT (p)/LT (f_i)$
\STATE $p:=p-(LT(p)/LT (f_i)) f_i$
\ELSE
\STATE $i:=i+1$
\ENDIF
\ENDWHILE
\IF{divisionoccured=false}
\STATE $r:=r+LT (p)$
\STATE $p:=p-LT (p) $
\ENDIF
\ENDWHILE
\end{algorithmic}
\end{algorithm}      %Probl�me typo 
\end{Thm}


%****************************************************************
\begin{demo}
Remarquons tout d'abord que lors de chaque it�ration de la boucle, une de ses deux instructions :
\begin{enumerate}
	\item Si $LT(f_i) | LT(p)$ alors on fait la division de $p$ par $f_i$
 \item Sinon on ajoute $LT(p)$ � $r$ (et on retire $LT(p)$ � $p$).
\end{enumerate}
Montrons d'abord que l'algorithme s'arr�te i.e. il existe une �tape o� $p=0$. \\
Pour cela, montrons que la suite des mon�mes dominants des diff�rentes valeurs $p$ est strictement d�croissante tant que $p \neq 0$. Si l'algorithme ne s'arr�tait pas, on aurait une suite infinie strictement croissante ce qui contredit le fait que $\geq$ est un bon ordre. \\

-Si on fait une division (par $f_j$) alors $p$ prend la valeur $p':=p-\frac{LT(p)}{LT(f_j)}f_j$. \\
-Si cette valeur est nulle alors l'algorithme s'arr�te sinon comme on a l'�galit� : $LT\left(\underbrace{\frac{LT(p)}{LT(f_j)}}_{\in k^*\mathscr{M}}f_j\right)=\frac{LT(p)}{LT(f_j)}LT(f_j)=LT(p)$.\\
On en d�duit donc,d'apr�s le lemme, que $LM(p') <LM(p)$. \\
-Sinon, $p$ prend la valeur $p-LT(p)$. Par le m�me argument que pr�c�demment, $LM(p-LT(p))<LT(p)$. \\
Ce qui permet de conclure. \\
Montrons maintenant qu'� chaque �tape que $f=\sum_{i=0}^s a_if_i +p+r$. \\
Initialisation de l'algorithme ("0�me it�ration") : Comme $a_1=\ldots=a_s=r=0$ et $p=f$ alors l'�galit� est v�rifi�e. \\
H�r�dit� : Soit $n \in \mathbb{N}$ et supposons qu'� la $n$�me it�ration de la boucle, $f=\sum_{i=0}^s a_if_i +p+r=\sum_{i=0,i \neq j}^s a_if_i +a_jf_j+p+r$ pour tout $j\in [\![1,n]\!]$  alors : \\
- si on fait une division ( $p$ avec $f_j$) alors : la nouvelle valeur $p'$ de $p$ est $p-\frac{LT(p)}{LT(f_j)}f_j$ et celle de $a_i$ est $a'_j=a_j+\frac{LT(p)}{LT(f_j)}$. et donc : \\
$\sum_{i=0,i \neq j}^s a_if_i +a'_jf_j+p'+r=\sum_{i=0,i \neq j}^s a_if_i +\left(a_j+\frac{LT(p)}{LT(f_j)}\right)f_j+p-\frac{LT(p)}{LT(f_j)}f_j+r$ \\
$=\sum_{i=0,i \neq j}^s a_if_i +a_jf_j+p+r=f$.
On obtient donc,lorsque $p=0$ (et on sait que cela arrivera), que  $f=\sum_{i=1}^s a_if_i+r$ et $r$ est, par d�finition dans l'algorithme, une somme d'�l�ments non divisibles par les $LT(f_i)$
\end{demo}