\begin{Def}
Soit $I=\left\langle f_1,\ldots,f_s \right\rangle $ un id�al de $k[X_1,\ldots,X_n]$. On appelle $k$�me id�al d'�limination de $I$ l'id�al $I_k$ de $k[X_{k+1},\ldots,X_n]$ d�finit par : $I_k=k[X_{k+1},\ldots,X_n] \cap I$
\end{Def}
 

\begin{Thm}[d'�limination]

Soit $I $ un id�al de $k[X_1,\ldots,X_n]$ et $G$ une base de Gr�bner de $I$ selon l'ordre lexicographique (que l'on notera ici seulement $\geq$). Alors, pour tout $k \in [\![0,n]\!]$, l'ensemble $G_k =G \cap k[X_{k+1},\ldots,X_n]  $ est une base de Gr�bner du $k$�me id�al d'�limination $I_k$.
\end{Thm}

\begin{demo}

Soit $k \in [\![0,n]\!]$. Posons $G=\{g_1,\ldots,g_m\}$ et tel que $G_k=\{g_1,\ldots,g_r\}$ (quitte � renommer les �l�ments). \\
Montrons que $G_k$ est une base de $I_k$. \\
Comme $G_k \subset I_k$ (car $G \subset I$) alors $\left \langle G_k \right\rangle \subset I_k$.\\
Soit $f \in I_k$ alors d'apr�s le th�or�me de division par $G$,il existe $h_1,\ldots,h_m \in k[X_1,\ldots,X_n]$,\\
$f=\sum_{k=1}^m h_ig_i$ car ($G$ est une base de Gr�bner de $I$ et $f \in I$)\\
or pour tout $k>r, g_i > X^{k+1} \geq LM(f) $ et donc aucun terme de $f$ ne peut �tre divisible par un $LT(g_i)$. L'algorithme n'incr�mente pas les $h_k$($k >r$) et donc sont tous nuls.  \\
D'o�, $f=\sum_{k=1}^r h_ig_i$ et donc $f_k \in \left \langle G_k \right\rangle$, ce qui finit de montrer l'�galit� $\left \langle G_k \right\rangle = I_k$.\\
(Le m�me argument permet de montrer que si $f \in I_k$, $\overline{f}^G=\overline{f}^{G_k}$ ).\\
Montrons maintenant que $G$ est une base de Gr�bner de $I_k$.\\
Il suffit, pour cela, de montrer que pour tout $1 \leq i < j \leq r$, $\overline{S(g_i,g_j)}^{G_k}=0$.\\
Soit $i,j \in [\![1,r]\!]$,$ i<j$.\\
Comme $S(g_i,g_j)$ est de la forme $Pg_i+Qg_j$ ($P,Q \in k[X_{k+1},\ldots,X_n]$) et $I_k$ est un id�al alors $S(g_i,g_j) \in I_k \subset I$ d'o� comme $G$ est une base de Gr�bner alors $\overline{S(g_i,g_j)}^{G}=0$ et donc d'apr�s la remarque pr�c�dente, \\
$\overline{S(g_i,g_j)}^{G_k}=0$. Ce qui permet de conclure.
\end{demo}
 

\begin{Thm}[d'extension]

 Soit $I=\left \langle f_1,\ldots,f_s \right\rangle$ un id�al de $\C[X_1,\ldots,X_n]$ et $I_1$ le premier id�al d'�limination.\\
Ecrivons, pour $i \in [\![1,s]\!]$, $f_i$ sous la forme \\
$f_i=g(X_2,\ldots,X_n)X_1^{N_i}$+termes de degr� $<N_i$ en $X_1$\\
o� $N_i \geq 0$ et $g_i \in \mathbb{C}[X_2,\ldots,X_n]$ non nul si $f_i \neq 0$ ($g_i=0$ si $f_i=0$). \\
Supposons qu'on ait une solution partiel $(a_2,\ldots,a_n) \in Z(I_1)$. Si $(a_2,\ldots,a_n) \notin Z(g_1,\ldots,g_s)$ alors il existe $a_1 \in \mathbb{C}$ tel que $(a_1,\ldots,a_n) \in Z(I)$.
\end{Thm}

\begin{cor}

 Soit $I=\left \langle f_1,\ldots,f_s \right\rangle$ un id�al de $\mathbb{C}[X_1,\ldots,X_n]$ et supposons qu'il existe $i \in [\![1,n]\!]$ tel que $f_i$ s'�crit de la forme \\
$f_i=cX_1^{N}$+termes de degr� $<N$ en $X_1$\\
o� $N> 0$ et $c \in \mathbb{C}\neq \{0\}$ non nul. Si $I_1$ est le premier id�al d'�limination de $I$ et $(a_2,\ldots,a_n) \in Z(I_1)$ alors il existe $a_1 \in \mathbb{C}$ tel que $(a_1,\ldots,a_n) \in Z(I)$
\end{cor}

\begin{demo}
Cons�quence imm�diate du th�or�me d'extension.\\
(Comme $g_i=c \neq 0$ alors $Z(g_1,\ldots,g_s) = \emptyset$ et donc $(a_2,\ldots,a_n) \notin Z(g_1,\ldots,g_s)$ pour tout $(a_2,\ldots,a_n) \in \C^{n-1}$).
\end{demo}
