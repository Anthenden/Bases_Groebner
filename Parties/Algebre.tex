\subsection{Polyn\^omes irr\'eductibles et factorisation}
 \begin{Def}
 Un polyn\^ome $P \in k[X_1,\ldots,X_n]$ est irr\'eductible sur $k$ si $P$ est non constant et qu'il n'est pas le produit de deux polyn\^omes non constants de $k[X_1,\ldots,X_n]$
 \end{Def} 
\begin{prop}
 Tout polyn\^ome non constant de $k[X_1,\ldots,X_n]$ peut s'\'ecrire comme produit de polyn\^omes irr\'eductible sur $k$. 
\end{prop}
 \begin{Thm} 
Soit $P \in k[X_1,\ldots,X_n]$ irr\'eductible sur $k$ et supposons que $P$ divise le produit $QR$, avec $Q,R \in k[X_1,\ldots,X_n]$. Alors $P$ divise $Q$ ou $R$. 
\end{Thm} 
\begin{Thm} 
 Tout polyn\^ome non constant $f\in k[X_1,\ldots,X_n]$ peut s'\'ecrire comme un produit $f=f_1 \ldots f_r$ d'irr\'eductible sur $k$. De plus, $f=g_1\ldots g_s$ est une autre factorisation en irr\'eductible sur $k$, alors $r=s$ et les $g_i$ peuvent être permut\'es de tel sorte que pour tout $i$,  $g_i$ soit un multiple de $f_i$ . 
\end{Thm}