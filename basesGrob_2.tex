\documentclass[a4page,10pt]{article}
\usepackage[Latin1]{inputenc}
\usepackage[francais]{babel}
\usepackage{amsmath,amssymb,amsthm}
\usepackage{textcomp}
\usepackage{mathrsfs}
\usepackage{ algorithm , algorithmic }
\usepackage[all]{xy}
\usepackage{hyperref}

%%% francisation des algorithmes

\makeatletter

\renewcommand*{\ALG@name}{Algorithme}

\makeatother

\renewcommand{\algorithmicrequire}{\textbf{\textsc {Entr�es  :  } } }
\renewcommand{\algorithmicensure}{\textbf{\textsc { Sortie  :  } } }
\renewcommand{\algorithmicwhile}{\textbf{Tant que}}
\renewcommand{\algorithmicdo}{\textbf{faire }}
\renewcommand{\algorithmicif}{\textbf{Si}}
\renewcommand{\algorithmicelse}{\textbf{Sinon}}
\renewcommand{\algorithmicthen}{\textbf{alors }}
\renewcommand{\algorithmicend}{\textbf{fin}}
\renewcommand{\algorithmicfor}{\textbf{Pour}}
\renewcommand{\algorithmicuntil}{\textbf{Jusqu'�}}
\renewcommand{\algorithmicrepeat}{\textbf{R�p�ter}}

\newcommand{\F}{\mathbb{F}}
\newcommand{\K}{\mathbb{K}}
\newcommand{\Z}{\mathbb{Z}}
\newcommand{\N}{\mathbb{N}}
\newcommand{\B}{\mathcal{B}}
\newcommand{\GL}{\mathcal{G}\mathcal{L}}
\newcommand{\SL}{\mathcal{S}\mathcal{L}}
\renewcommand{\L}{\mathcal{L}}
\newcommand{\R}{\mathbb{R}}
\newcommand{\C}{\mathbb{C}}
\newcommand{\Ccal}{\mathcal{C}}
\newcommand{\Rcal}{\mathcal{R}}
\newcommand{\Ecal}{\mathcal{E}}
\newcommand{\Fcal}{\mathcal{F}}
\newcommand{\Mcal}{\mathcal{M}}
\newcommand{\Acal}{\mathcal{A}}
\newcommand{\Lcal}{\mathcal{L}}
\newcommand{\Rscr}{\mathscr{R}}
\newcommand{\Ebar}{\overline{E}}
\newcommand{\xbar}{\overline{x}}
\newcommand{\ybar}{\overline{y}}
\newcommand{\fbar}{\overline{f}}
\newcommand\bigzero{\makebox(0,0){\text{\huge0}}}

\DeclareMathOperator{\multideg}{\mathrm{multideg}}
\DeclareMathOperator{\LM}{\mathrm{LM}}
\DeclareMathOperator{\LT}{\mathrm{LT}}
\DeclareMathOperator{\LC}{\mathrm{LC}}
\DeclareMathOperator{\PPCM}{\mathrm{PPCM}}
\DeclareMathOperator{\Syl}{\mathrm{Syl}}
\DeclareMathOperator{\Res}{\mathrm{Res}}
\DeclareMathOperator{\Com}{\mathrm{Com}}
\headheight=0mm
\topmargin=-10mm
\oddsidemargin=-1cm
\evensidemargin=-1cm
\textwidth=18cm
\textheight=24cm
\parindent=0mm
\begin{document}
\title{Bases de Gr�bner}
\author{Antoine BOIVIN}
\maketitle


\newtheorem{Thm}{Th�or�me}[section]
\newtheorem{prop}[Thm]{Propri�t�}
\newtheorem{Lemme}[Thm]{Lemme}
\newtheorem{cor}[Thm]{Corollaire}

\theoremstyle{definition}

\newtheorem{Ex}[Thm]{Exemple}
\newtheorem{Def}[Thm]{D�finition}
\newtheorem{Not}[Thm]{Notation}
\theoremstyle{remark}
\newtheorem{Rq}[Thm]{Remarque}


%\section*{Introduction}
La notion de base de Gr�bner d'un id�al d'un anneau de polyn�mes a �t� introduite de mani�re ind�pendante par Heisuke Hironaka (qu'il nomma "bases standard") et un peu plus tard par Bruno Buchberger (qu'il nomma "Bases de Gr�bner" en l'honneur de son directeur de th�se Wolgang Gr�bner).

Une base de Gr�bner d'un id�al est une famille de "bons" g�n�rateurs de cet id�al afin de d�crire plus facilement celui-ci.

Dans ce m�moire, nous allons, � travers l'�tude des bases de Gr�bner, traiter les quatre probl�mes suivants : 
\begin{enumerate}
\item  La description d'un id�al : Peut-on trouver une famille g�n�ratrice pour chaque id�al $I \subset k[X_1,\ldots,X_n]$?
\item L'appartenance � un id�al : Soit $f \in k[X_1,\ldots,X_n]$ et un id�al $I:=\left \langle f_1,\ldots,f_s \right \rangle$. Comment montrer que $f \in I$ ?
 \item La r�solution de syst�mes polynomiaux : Trouver les z�ros communs dans $k^n$ d'une famille de polyn�mes.
 \item Le probl�me d'implicitation : Soit $V$  le sous-ensemble de $k^n $ param�tr� par : $x_i=g_i(t_1,\ldots,t_n) , 1 \leq i \leq n $ o� les $g_i$ est des polyn�mes ou des fractions rationelles.
Peut-on trouver des �quations polynomiales en $X_i$ d�crivant $V$ ?
\end{enumerate}

En premier lieu, nous allons introduire la notion de base de Gr�bner et r�pondre aux probl�mes 1 et 2 gr�ce, notamment,  � un algorithme de division g�n�ralisant l'algorithme d'Euclide pour les polyn�mes � une ind�termin�e.
Nous allons ensuite discuter de l'�limination des variables dans les syst�mes polynomiaux (afin de r�soudre le probl�me 3) et de sa g�om�trie. Pour finir, nous allons utiliser ses r�sultats pour trouver les �quations d�crivant un ensemble dont on a la param�trisation.

\section*{Conventions et Notations}
Dans ce rapport, $k$ est un corps commutatif de caract�ristique 0 et $n$ un entier naturel strictement positif. \\
On consid�rera l'anneau de polyn�mes $k[X_1,\ldots,X_n]$ � $n$ ind�termin�es $X_1,\ldots,X_n$ construit par r�currence par : $\forall k<n, k[X_1,\ldots,X_{k+1}]:=k[X_1,\ldots,X_k][X_{k+1}]$. On pourra identifier cet anneau � celui des fonctions polyn�miales � $n$ variables (gr�ce � l'infinit� de $k$). \\
On appelle mon�mes tout polyn�me de la forme $X_1^{\alpha_1}\ldots X_n^{\alpha_n}$ o� $\alpha_i \in \N$ que l'on abr�gera $X^{(\alpha_1,\ldots,\alpha_n)}$. On notera $\mathscr{M}$ l'ensemble des mon�mes. Pour tout $p \in \N$ et pour tout anneau $A$, on appelera $A_p[X]$ l'ensemble des polyn�mes de degr� strictement inf�rieur � $p$.\\
On appelera base d'un id�al $I$ une famille $F=(f_1,\ldots,f_s)$ telle que $I=\left\langle F\right\rangle$.

%\tableofcontents
\newpage


\section{Motivation}
\subsection{Exemple introductif}
On va commencer par traiter le cas simple de l'anneau des polyn�mes � un ind�termin�e $k[X]$. \\
$k[X]$ est un anneau euclidien et donc principal. De ce fait, on a un description simple des id�aux de $k[X]$ qui sont tous engendr�s par un unique �l�ment. \\
De plus, gr�ce � l'algorithme d'Euclide, on peut ais�ment v�rifier l'appartenance d'un polyn�me � un id�al de $k[X]$. \\
Soient $I=\left\langle P\right\rangle \subset k[X]$ et $f \in k[X]$. \\
En effectuant la division euclidienne de $f$ par $P$, on a : $f=PQ+R$ o� $\deg(R) < \deg(P)$. \\
On a alors la caract�risation suivante :  $f \in I \Leftrightarrow R=0$. \\
Ce qui r�gle les probl�mes 1 et 2 de l'introduction pour $k[X]$.\\

Cependant, lorsque $n>1$, $k[X_1,\ldots,X_n]$ n'est plus euclidien et m�me plus principal. Nous allons montrer ce r�sultat et d�finir la notion plus faible d'anneau noeth�rien qui convient � $k[X_1,\ldots,X_n]$

\subsection{$k[X_1,\ldots,X_n]$ n'est pas principal}
Pour montrer que $k[X_1,\ldots,X_n]$ n'est pas principal, nous allons montrer deux lemmes : le premier qui est la r�ciproque � la principalit� de $k[X]$ et le second pour faire marcher l'autre lemme dans une r�currence sur le nombre d'ind�termin�es.
\begin{Lemme}
Soit $A$ un anneau int�gre commutatif.\\
$A[X]$ est principal si, et seulement si, $A$ est un corps. 
\end{Lemme}
\begin{proof} 
$\Leftarrow$  : $A[X] $ est euclidien pour le stathme euclidien du degr�. Comme $A[X]$ est aussi int�gre alors $A[X]$ est principal.\\
$\Rightarrow$ : Soit  $a \in A\setminus\{0\}$.\\
Soit $I=\left\langle a,X \right\rangle=\left\langle P \right\rangle$ car $A[X]$ est principal.\\
Comme $P$ divise $a$ et que $A[X]$ est int�gre(car $A$ est int�gre) alors $P$ est constant (degr�).\\
De plus, comme $P$ divise $X$ alors il existe $a,b \in A$ tel que $X=P(aX+b)$ (degr�).\\
D'o� $X=aPX+bP$ ou encore, par unicit� des coefficients, $aP=1$ et donc $a$ est inversible.
\end{proof}
\begin{Lemme}
Soit $A$ un anneau int�gre commutatif. $A[X]$ n'est pas un corps.
\end{Lemme}
\begin{proof} 
$X$ n'a pas d'inverse pour des questions de degr� (par int�grit� de $A$).
\end{proof}
\begin{Thm}
$k[X_1,\ldots,X_n]$ n'est pas principal pour $n>1$.
\end{Thm}
\begin{proof}
On le montre par r�currence sur le nombre d'ind�termin�es : \\
En utilisant les deux lemmes pr�c�dents, on obtient que $ k[X_1,X_2]=k[X_1][X_2]$ n'est pas principal car $k[X_1]$ n'est pas un corps.\\
Soit $n>1$ et supposons que $k[X_1,\ldots,X_n]$ n'est pas principal (et \emph{ a fortiori } n'est pas un corps).\\
$k[X_1,\ldots,X_{n+1}]=k[X_1,\ldots,X_n][X_{n+1}]$ n'est pas principal par le lemme $1$.
\end{proof}
\subsection{$k[X_1,\ldots,X_n]$ est noeth�rien}

\begin{Def}
Soit $A$ un anneau commutatif. On dit que $A$ est un anneau noeth�rien si tout id�al $I$ de $A$ est engendr� par un nombre fini d'�l�ments.
\end{Def}
Dans la suite, on aura aussi besoin d'une autre caract�risation des anneaux noeth�riens
\begin{Rq}
Une suite croissante d'id�aux d'un anneau noeth�rien est stationnaire.
\end{Rq}
\begin{Ex} Tout anneau principal est noeth�rien car chaque id�al d'un anneau principal $A$  est de la forme $aA$ o� $a \in A$.\\
En particulier, tout corps est noeth�rien (les id�aux d'un corps sont $\{0\}$ et lui-m�me).
\end{Ex}

Maintenant que nous avons la d�finition d'anneau noeth�rien, nous allons montrer que $k[X_1,\ldots,X_n]$ est noeth�rien gr�ce au th�or�me de transfert suivant : 
\begin{Thm}[de la base de Hilbert]

Soit $A$ un anneau noeth�rien. Alors $A[X]$ est aussi un anneau noeth�rien.
\end{Thm}

\begin{proof}

Soient $I$ un id�al de $A[X]$. Soit $J$ l'id�al engendr� par les coefficients dominants des polyn�mes de $A[X]$ et pour tout $n \in \mathbb{N}$, $J_n$ l'id�al engendr� par les coefficient en $X^n$ de $A_{n+1}[X]$, deux id�aux de $A$.\\
Comme $A$ est noeth�rien alors il existe $x_1,\ldots,x_r \in I$,tels que $J=\left\langle x_1,\ldots,x_r\right\rangle$ et pour tout $n \in \mathbb{N}, y_{1,n},\ldots,y_{m_n,n}$ tels que $J_n=\left\langle y_{1,n},\ldots,y_{m_n,n} \right\rangle$.\\
Il existe donc des polyn�mes $Q_1,\ldots,Q_r$ de $I$ ayant pour coefficient dominant $x_i$ et pour tout $n \in \mathbb{N}$, des polyn�mes $R_{1,n},\ldots R_{m_n,n} \in I \cap A_n[X] $ dont le coefficient en $X^n$ est $y_{m_n,n}$.\\
Montrons que $I=\left\langle  Q_1,\ldots,Q_r, R_{1,1},\ldots,R_{m_1,1},\ldots,R_{1,N},\ldots,R_{m_N,N} \right \rangle$ o� $N:=\max_i \; deg(Q_i)$.\\
Notons $I'$  cet id�al (inclus dans $I$ car engendr� par des �l�ments de $I$). \\
Pour montrer l'inclusion r�ciproque, il suffit de montrer que, pour tout $d \in \mathbb{N}$, $I \cap A_d[X] \subset I'$
Montrons donc,par r�currence, la propri�t� suivante : \\$\forall d \in \mathbb{N}, \forall P \in I, deg(P) <d \Rightarrow P \in I'$

\textbf{Initialisation }: Si $n=0$, $I \cap A_n[X]=\{0\} \subset I' $ (car $0$ est le seul polyn�me de degr� strictement n�gatif et $0$ est dans tout id�al de $A[X]$)

\textbf{H�r�dit�} : Soit $n \in \mathbb{N}$ et supposons que pour tout polyn�mes $P$ de $I$ de degr� strictement inf�rieur � $n$ appartient � $I$.\\
Soit $P:=\sum_{k=0}^n a_k X^k \in I$.
\begin{itemize}
\item Si $d \leq N-1$ alors $a_d \in J_d$, il existe donc $\lambda_1,\ldots,\lambda_{m_d}$ tel que $a_d=\sum_{k=1}^{m_d} \lambda_k y_{k,d}$. On en d�duit que $T:=P-\sum_{k=1}^{m_d} \lambda_k R_{k,d}$ est de degr� inf�rieur � $d-1$. Comme $P$ et les $R_{k,d}$ sont dans $I$ alors $T$ aussi et par hypoth�se de r�currence $T \in I$. Comme les $R_{k,d}$ sont aussi dans $I'$ alors $P=T+\sum_{k=1}^{m_d} \lambda_k R_{k,d}$ est dans $I'$.
\item Si $d \geq N$, alors $a_d \in J$,il existe donc $\lambda_1,\ldots,\lambda_r \in A$ tel que $a=\sum_{k=1}^r \lambda_k x_k$ et donc $P-\sum_{k=1}^m \lambda_i X^{d-deg(Q_i)} Q_i$ est de degr� inf�rieur � $d-1$. On en d�duit de la m�me fa�on que dans le cas pr�c�dent que $P\in I'$.
\end{itemize}
Conclusion : D'apr�s le principe de r�currence, pour tout $d \in \mathbb{N}$, $I \cap A_d[X] \subset I'$.\\
On en d�duit que $I \subset I'$  et donc $I=I'$. 
Ce qui nous permet de conclure que $I$ est engendr� par un nombre fini d'�l�ments. \\
Ce qui montre que $A[X]$ est donc noeth�rien.
\end{proof}
\begin{cor}
$k[X_1,\ldots,X_n]$ est un anneau noeth�rien.
\end{cor}
Ce qui cl�t le probl�me 1 de l'introduction : la description des id�aux de $k[X_1,\ldots,X_n]$
\section{Algorithme de division}
Nous avons rappel�, dans la premi�re section, que l'on a la caract�risation suivante dans $k[X]$ :
pour tout id�al $I:=\left\langle P \right\rangle$ et pour tout $f \in k[X]$, on peut �crire $f$ sous la forme $QP+R$ (gr�ce � l'algorithme d'Euclide) o� $\deg R < \deg P$ et $f \in I \Leftrightarrow R=0$. \\

Pour le cas g�n�ral, pour montrer que le polyn�me $f$ de $k[X_1,\ldots,X_n]$ appartient � un id�al $I:=\left\langle f_1,\ldots,f_s \right\rangle$, nous allons d�crire un algorithme permettant d'�crire $f$ sous la forme  $\sum g_if_i+r$. Avant cela,en analogie � l'algorithme en une ind�termin�e, il nous faut d�finir un ordre, afin qu'� chaque �tape, on puisse dire que le "terme dominant" du reste est strictement inf�rieur � celui d'avant (celui sur $k[X]$ est $X^n \geq X^m \Leftrightarrow n \geq m$). Cet ordre doit �tre, de plus, un bon ordre pour ne pas avoir de suites infinies strictement d�croissantes et donc pour que l'algorithme finisse.
\subsection{Ordres monomiaux}
\subsubsection{D�finitions}
\begin{Def} 
Un ordre monomial est une relation d'ordre total $\geq$ de $\mathscr{M}$ telle que :

\begin{enumerate}
	\item  $\forall \alpha,\beta,\gamma \in \mathbb{N}^n, X^\alpha \geq X^\beta \Rightarrow X^{\alpha+\gamma} \geq X^{\beta +\gamma}$(compatibilit� avec le produit)
 \item $\geq$ est un bon ordre
\end{enumerate}
On note $X^\alpha > X^\beta$ si $X^\alpha \geq X^\beta$ et $\alpha \neq \beta$  et $X^\alpha \leq X^\beta $ si $X^\beta \geq X^\alpha $
\end{Def}
On peut remarquer que l'ordre induit par le degr� dans $k[X]$ est un ordre monomial.

\begin{Def}
Soit $P:=\sum_{\alpha} p_{\alpha}X^\alpha \in k[X_1,\ldots,X_n]$ et $\geq$ un ordre monomial. \\
\begin{enumerate}
	\item  Le mon�me dominant de $P$ est : $\LM(P):=\max\{X^\alpha \in \mathscr{M} | p_\alpha \neq 0\}$
\item Le multidegr� de $f$ est l'�l�ment de $\mathbb{N}^n$, not� $\multideg(P)$, tel que $x^{\multideg(P)}=LM(P)$
 \item Le coefficient dominant de $P$ est $\LC(P):=p_{\multideg(P)}$
\item Le terme dominant de $P$ est $\LT(P):=\LC(P)\cdot \LM(P)$
\end{enumerate}

\end{Def}
\subsubsection{Exemples d'ordres monomiaux}
Nous allons maintenant donner quelques exemples d'ordres monomiaux et en premier lieu, l'ordre lexicographique que nous allons utiliser pour l'�limination.

\begin{Def}[Ordre lexicographique $\geq_{lex}$]
Soient $\alpha=(\alpha_1,\ldots,\alpha_n), \beta=(\beta_1,\ldots,\beta_n) \in \mathbb{N}^n$ alors
$X^\alpha \geq_{lex} X^\beta$ si, et seulement si, $\alpha=\beta$ ou le premier coefficient non nul en lisant par la gauche de $\alpha -\beta $ est positif.
\end{Def}
\begin{prop}
$\geq_{lex}$ est un ordre monomial.
\end{prop}
\begin{proof}
Montrons, par l'absurde, que $\geq_{lex}$ est un bon ordre. \\
Supposons donc que  $\geq_{lex}$ n'est pas un bon ordre et donc qu'il existe une suite $u:=\left(X^{(a_{1,i}\ldots,a_{n,i})}\right)_{i \in \mathbb{N}}$ strictement d�croissante.\\
On en d�duit que la suite $u_1:=(a_{1,i})_{i \in \mathbb{N}}$ est d�croissante (sinon $u$ ne serait pas d�croissante) et est donc stationnaire car $\mathbb{N}$ est bien ordonn�.\\
Alors il existe $N_1 \in \mathbb{N}$ tel que $\forall p \geq N,u_{1,p}=u_{1,N_1}$. \\
Consid�rons maintenant la suite $u_2:=(a_{2,i})_{i \geq N_1}$. Elle est d�croissante et donc stationnaire ... \\
On construit ainsi une suite $(N_i)_{i \geq 1}$ telle que : $\forall i \in \N^*,\forall n \geq N_i, u_{i,n} = u_{i,N_i}$.\\
On en d�duit que $\forall p \geq N_n,\forall i \in [\![1,n]\!], u_{i,p}=u_{i,N_n}$ ou encore \\$\forall p \geq N_n, X^{u_{1,p},\ldots,u_{n,p}}=X^{u_{1,N_n},\ldots,u_{n,N_n}}$, ce qui est contradictoire avec la stricte d�croissance de $u$.
\end{proof}
Les deux ordres suivants comparent les degr�s totaux des mon�mes.
\begin{Def}
Le degr� total du mon�me $X_1^{\alpha_1}\ldots X_n^{\alpha_n}$ est $|\alpha_1,\ldots,\alpha_n|:=\sum \alpha_i$\\
Le degr� d'un polyn�me $f \in k[X_1,\ldots,X_n]$ est le maximum des degr�s totaux des mon�mes le constituant.

\end{Def}
\begin{Def}[Ordre lexicographique gradu� $\geq_{grlex}$]
Soient $\alpha=(\alpha_1,\ldots,\alpha_n), \beta=(\beta_1,\ldots,\beta_n) \in \mathbb{N}^n$ alors
$X^\alpha \geq_{grlex} X^\beta$ si, et seulement si, $|\alpha|>|\beta|$ ou ($|\alpha|=|\beta|$ et $\alpha\geq_{lex} \beta$).
\end{Def}
\begin{Def}[Ordre lexicographique gradu� renvers� $\geq_{grevlex}$]
Soient $\alpha=(\alpha_1,\ldots,\alpha_n), \beta=(\beta_1,\ldots,\beta_n) \in \mathbb{N}^n$ alors
$X^\alpha \geq_{grevlex} X^\beta$ si, et seulement si, $|\alpha|>|\beta|$ ou ($|\alpha|=|\beta|$ et le premier coefficient non nul en lisant par la droite de $\beta - \alpha $ est positif).
\end{Def}
La diff�rence entre ses deux ordres sont dans la mani�re o� ils d�cident des �galit�s des mon�mes de m�me degr� total : le premier se base sur l'ordre lexicographique et le second sur un ordre lexicographique "renvers�", dans le sens o� il commence par la droite et regarde la premi�re entr�e n�gative de $\alpha-\beta$.
\begin{Ex}
On va comparer les polyn�mes $f=X^2Z^2$,$g=X^2Z$ et $h=XY^2Z$ de  $k[X,Y,Z]$ avec ces trois diff�rents ordres : \\
$f>_{lex}g >_{lex}h$ \\
$f>_{grlex}h >_{grlex}g$ \\
$h>_{grevlex}f >_{grevlex}g$ \\
\end{Ex}

\subsection{Algorithme de division}
Dans ce paragraphe, nous allons �noncer et d�montrer l'algorithme de division.
  \begin{Lemme} Soit $f,g \in k[X_1,\ldots,X_n]$ tels que $\LT(f)=\LT(g)$ alors $\LM(f-g) <\LM(f)=\LM(g)$
	\label{LM_dec}
	\end{Lemme}
	\begin{proof}
	Soit $\alpha,\alpha_1,\ldots,\alpha_n \in \mathbb{N}^n $ tel que : $X^\alpha >X^{\alpha_1}>\ldots >X^{\alpha_n}$ et $f=pX^\alpha+\sum p_{\alpha_i}X^{\alpha_i}$ et $g=pX^\alpha+\sum q_{\alpha_i}X^{\alpha_i}$ alors $\LM(f-g)=\LM(\sum (p_{\alpha_i}-q_{\alpha_i})X^{\alpha_i}) \leq X^{\alpha_1} < X^\alpha=\LM(f)=\LM(g)$\end{proof}
	\begin{Lemme} Soit $f \in k[X_1,\ldots,X_n]$, $\alpha \in \N^n$ et $p \in k^*$.
	Alors $\LT(pX^\alpha f)=pX^\alpha \LT(f)$
	\label{LT_prod}
	\end{Lemme}
	\begin{proof}
	Soit $f = \sum a_lX^l$ et posons $\beta:=\LM(f)$. \\
	Alors $\LT(pf)=p a_\beta X^\beta=p\LT(f)$ et $\LT(X^\alpha f)=a_\beta X^{\beta+\alpha}=X^\alpha\LT(f)$ car si pour tout $l \in \N^n, X^l \leq X^\beta$ alors, par compatibilit� de l'ordre monomial avec le produit,$\forall l \in \N^n, X^{l + \alpha} \leq X^{\alpha+\beta}$ 
	
	\end{proof}



\begin{Thm}[Algorithme de division]

Soit $\geq$ un ordre monomial et $(f_1,\ldots,f_s)$ un $n$-uplet de polyn�me de $k[X_1,\ldots,X_n]$. Alors tout polyn�me $f$ de $k[X_1,\ldots,X_n]$ peut s'�crire sous la forme $f=\sum_{i=1}^n a_if_i+r$ o� $a_i \in k[X_1,\ldots,X_n]$ et $r$ est une combinaison lin�aire de mon�mes qui ne sont pas divisibles par les $\LT(f_i)$
   

\end{Thm}
  
\begin{algorithm}

\caption{Algorithme de division}

\begin{algorithmic}
\REQUIRE $f_1,\ldots,f_s,f$
\ENSURE $a_1,\ldots,a_s,r$
\STATE $a_1 :=0 ;\ldots ; a_s:=0 ; r:=0$
\STATE $p:=0$
\WHILE{$p \neq 0$}
\STATE $i:=1$
\STATE $divisionoccured:=false $
\WHILE{$i \leq s $ et $divisionoccured = false $}
\IF{$\LT (f_i) | \LT (p)$}
\STATE $a_i:= a_i+\LT (p)/\LT (f_i)$
\STATE $p:=p-(\LT(p)/\LT (f_i)) f_i$
\ELSE
\STATE $i:=i+1$
\ENDIF
\ENDWHILE
\IF{divisionoccured=false}
\STATE $r:=r+\LT (p)$
\STATE $p:=p-\LT (p) $
\ENDIF
\ENDWHILE
\end{algorithmic}

\end{algorithm} 
\begin{proof}
Pour montrer ce th�or�me, nous allons que l'algorithme ci-dessus donne le bon r�sultat et finit pour tous polyn�mes mis en entr�e du polyn�me.\\

Remarquons tout d'abord que lors de chaque it�ration de la boucle, une de ses deux instructions est ex�cut�e :
\begin{enumerate}
	\item Si $\LT(f_i) | \LT(p)$ alors on fait la division de $p$ par $f_i$
 \item Sinon on ajoute $\LT(p)$ � $r$ (et on retire $\LT(p)$ � $p$).
\end{enumerate}
Montrons d'abord que l'algorithme s'arr�te i.e. il existe une �tape o� $p=0$. \\
Pour cela, montrons que la suite des mon�mes dominants des diff�rentes valeurs de $p$ est strictement d�croissante tant que $p \neq 0$. Si l'algorithme ne s'arr�tait pas, on aurait alors une suite infinie strictement croissante ce qui contredirait le fait que $\geq$ soit un bon ordre. \\

-Si on fait une division (par $f_j$) alors $p$ prend la valeur $p':=p-\frac{\LT(p)}{\LT(f_j)}f_j$. \\
Si cette valeur est nulle alors l'algorithme s'arr�te sinon, gr�ce � l'�galit� obtenue avec le lemme \ref{LT_prod} : \\$\LT\left(\underbrace{\frac{\LT(p)}{\LT(f_j)}}_{\in k^*\mathscr{M}}f_j\right)=\frac{\LT(p)}{\LT(f_j)}\LT(f_j)=\LT(p)$ \\alors, d'apr�s le lemme \ref{LM_dec}, on a que $LM(p') <LM(p)$. \\
-Sinon, $p$ prend la valeur $p-\LT(p)$. Par le m�me argument que pr�c�demment, $LM(p-\LT(p))<\LT(p)$. \\
Ce qui permet de conclure. \\
Montrons maintenant qu'� chaque �tape que $f=\sum_{i=0}^s a_if_i +p+r$. \\
Initialisation de l'algorithme ("0�me it�ration") : Comme $a_1=\ldots=a_s=r=0$ et $p=f$ alors l'�galit� est v�rifi�e. \\
H�r�dit� : Soit $n \in \mathbb{N}$ et supposons qu'� la $n$�me it�ration de la boucle, $f=\sum_{i=0}^s a_if_i +p+r=\sum_{i=0,i \neq j}^s a_if_i +a_jf_j+p+r$ pour tout $j\in [\![1,n]\!]$  alors : \\
- si on fait une division ( $p$ avec $f_j$) alors : la nouvelle valeur $p'$ de $p$ est $p-\frac{\LT(p)}{\LT(f_j)}f_j$ et celle de $a_i$ est $a'_j=a_j+\frac{\LT(p)}{\LT(f_j)}$. et donc : \\
$\sum_{i=0,i \neq j}^s a_if_i +a'_jf_j+p'+r=\sum_{i=0,i \neq j}^s a_if_i +\left(a_j+\frac{\LT(p)}{\LT(f_j)}\right)f_j+p-\frac{\LT(p)}{\LT(f_j)}f_j+r$ \\
$=\sum_{i=0,i \neq j}^s a_if_i +a_jf_j+p+r=f$.\\
- sinon, $f=\sum_{i=0,i \neq j}^s a_if_i +a_jf_j+p+r=\sum_{i=0,i \neq j}^s a_if_i +a_jf_j+(p-\LT(p))+(r+\LT(p))$. \\
On finit par obtenir que,lorsque $p=0$ (et on sait que cela arrivera),  $f=\sum_{i=1}^s a_if_i+r$ o� $r$ est, par d�finition, une somme d'�l�ments non divisibles par les $\LT(f_i)$
\end{proof}
Cet algorithme ne permet pas de savoir, en g�n�ral, si un polyn�me appartient � un id�al. En effet, le reste d�pend du $s$-uplet choisi et dans quel ordre sont les polyn�mes de celui-ci, comme l'illustre cet exemple.
\begin{Ex}
\label{ex}
Soient $f=X^2+Y^2+Z^2-4$, $g=Y^2+2*Z^2-5$ et $h=XY-1$ trois polyn�mes de $k[X,Y,Z]$ et appliquons au polyn�me $P=YZf-Yg+2Zh$ l'algorithme de division pour l'ordre lexicographique.
\begin{itemize}
	\item Avec la famille $(f,g,h)$, on obtient $P=2Zf+(XZ-Y)g+(-2XZ^3+5XZ-YZ)$
	\item Avec la famille $(h,g,f)$, on obtient $P=YZf-Yg+2Zh$
\end{itemize}
Ce ph�nom�ne n'existe pas lorsque les $f_i$ forment une "base de Gr�bner" que l'on �tudiera dans la section prochaine.
\end{Ex}


\section{Bases de Gr�bner}
Dans cette section, nous allons commencer par d�finir et donner quelques propri�t�s des bases de Gr�bner. Puis nous allons �noncer l'algorithme de Buchberger permettant de cr�er une base de Gr�bner � partir d'une base d'un id�al.
\subsection{G�n�ralit�s}
\begin{Not}
Soit $I$ un id�al de $k[X_1,\ldots,X_n]$ non nul. \\
$LT(I)$ est l'ensemble des termes dominants des polyn�mes de $I$ i.e. $\LT(I):=\{\LT(f) |f \in I \}$.
\end{Not}
\begin{Def}
Soit $\geq$ un ordre monomial. Un sous-ensemble $G=\{g_1,\ldots,g_s\}$ d'un id�al $I$ est une base de Gr�bner si $\left\langle \LT(I) \right\rangle=\left\langle \LT(g_1),\ldots,\LT(g_s)\right\rangle$.
\end{Def}

On va commencer par montrer qu'une base de Gr�bner de $I$ est effectivement une base de $I$.
\begin{Lemme}
\label{lem_div}
Soit $I:=\left\langle X^{\alpha} | \alpha \in A \right\rangle$ un id�al de $k[X_1,\ldots,X_n]$. Alors $X^\beta \in I$ si, et seulement si, il existe un $\alpha \in A$ tel que $X^\alpha $divise $X^\beta$.
\end{Lemme}

\begin{proof}
$\Rightarrow $ Si $X^\beta \in I$ alors il existe une famille de polyn�mes $P_1,\ldots,P_s \in k[X_1,\ldots,X_n]$ et d'exposants $\alpha_1,\ldots,\alpha_s \in A$ telle que $X^\beta = \sum_{i=1}^s P_i X^{\alpha_i}$. \\
On peut alors remarquer, en utilisant les expressions $P_i:=\sum p_{i,\alpha} X^\alpha$, que $X^\beta$ est de la forme $\sum_{\gamma \in \Gamma} p_\gamma X^{\gamma}$ o� $\Gamma:=\{\gamma \in \mathbb{N}^n | \exists n \in \mathbb{N}^n,\exists i \in [\![1,s]\!],\gamma=\alpha_i+n\}$. 
Et donc $X^\beta- \sum_{\gamma \in \Gamma} p_\gamma X^{\gamma}=0 \; (*)$ \\
Comme $k[X_1,\ldots,X_n]$ est un $k$-espace vectoriel de base canonique $\mathscr{M}$, on d�duit de $(*)$ que $p_\gamma=\begin{cases}
 0 \text{ si } \gamma \neq \beta \\
 1 \text{ sinon }
\end{cases}$ 
(dans le cas contraire, on aurait une combinaison lin�aire (d'�l�ment d'une base) nulle � coefficients non nuls). On en d�duit que $\beta \in \Gamma$ et donc qu'il existe un $p \in \mathbb{N}^n$ et un  $i \in [\![1,s]\!]$, $\beta=a_i+p $ c'est-�-dire qu'il existe un $i \in [\![1,s]\!]$ tel que $X^{\alpha_i}$ divise $X^\beta$.
\end{proof}
\begin{prop}
 Soit $\geq$ un ordre monomial. Alors tout id�al de $k[X_1,\ldots,X_n]$ non r�duit � $\{0\}$ a une base de Gr�bner. De plus, tout base de Gr�bner est une base de $I$.
\end{prop}

\begin{proof}
 Soit $I$ un id�al de $k[X_1,\ldots,X_n]$ non r�duit � $\{0\}$ et $g_1,\ldots,g_s \in I$ tels que $\left\langle \LT(I)\right\rangle=\left\langle \LT(g_1),\ldots,\LT(g_s)\right\rangle$. 
Comme les $g_i$ appartiennent � $I$ alors $I':=\LT(g_1),\ldots,\LT(g_s) \subset I$. Montrons l'inclusion r�ciproque : \\
Soit $f \in I$. Si $f=0$ alors, comme $I'$ est un id�al, $f \in I'$. \\
Supposons donc $f \neq 0$.
Comme $\LT(f) \in \left\langle \LT(I) \right\rangle$ alors il existe un $i$ tel que $\LT(g_i)$ divise $\LT(f)$ (d'apr�s le lemme \ref{lem_div}). \\
Alors $f-\frac{\LT(f)}{\LT(g_i)}g_i \in I$ (car $f$ et $g_i$ sont dans $I$).\\
En it�rant ce raisonnement, on obtient que le reste de la division de $f$ par $g_1,\ldots,g_s$ est nul car tous les termes dominants de polyn�me de $I$ sont divisibles par un $LT(g_i)$. On en d�duit que $f$ est de la forme $\sum g_ih_i$ et donc $f\in I'$.

\end{proof}

Nous allons maintenant montrer la propri�t� des bases de Gr�bner promise � la fin de la section pr�c�dente : l'unicit� du reste de la division par une base de Gr�bner.

\begin{prop}
 Soit $G=\{g_1,\ldots,g_s\}$ une base de Gr�bner d'un id�al $I$ de $k[X_1,\ldots,X_n]$ et $f \in I$. Alors il existe un unique $r \in k[X_1,\ldots,X_n]$ v�rifiant : 
\begin{enumerate}
	\item  Tous les termes de $r$ ne sont divisible par aucun des $\LT(g_i)$
\item Il existe $g \in I$ tel que $f=g+r$
\end{enumerate}
\end{prop}
\begin{proof}
 L'algorithme de division nous donne l'existence d'un tel $r$. Montrons son unicit�. \\
Supposons, par l'absurde, l'existence de deux restes $r_1$ et $r_2$, $r_1 \neq r_2$ v�rifiant $(1)$ et $(2)$. \\
Alors : $\begin{cases} f=g_1+r_1 \\f=g_2+r_2 \end{cases}$ et donc $r_1-r_2=g_1-g_2 \in I$. \\
D'o�, comme $r_1 \neq r_2$ alors $LT(r_1-r_2) \in \left\langle LT(I) \right\rangle=\left\langle g_1,\ldots,g_s \right\rangle$ et donc $LT(r_1-r_2)$ est divis� par un des $LT(g_i)$ (cf Lemme \ref{lem_div}). On obtient donc une contradiction car aucun terme de $r_1$ et $r_2$ n'est divisible par des $LT(g_i)$. D'o� $r_1=r_2$. \\
\end{proof}

\begin{cor} Soit $G=\{g_1,\ldots,g_s\}$ une base de Gr�bner d'un id�al $I$ de $k[X_,\ldots,X_n]$ et $f \in k[X_1,\ldots,X_n]$. Alors $f \in I$ si, et seulement si, le reste de la division de $f$ par $G$ est nul.
\end{cor}
 
Ce corollaire nous permet de r�pondre au probl�me 2 de l'introduction : l'appartenance d'un polyn�me � un id�al donn�. Maintenant pour pouvoir appliquer ce corollaire, on va �noncer une caract�risation des bases de Gr�bner plus maniable que celle de la d�finition.

\begin{Not} On notera $\overline{f}^G$ le reste de $f$ par la base de Gr�bner $G$.
\end{Not}

\begin{Def} Soient $f,g \in k[X_1,\ldots,X_n]$ des polyn�mes non nuls.
\begin{enumerate}
\item Posons $\multideg(f)=(\alpha_1,\ldots,\alpha_n)$ et $\multideg(g)=(\beta_1,\ldots,\beta_n)$. Soit $\gamma=(\gamma_1,\ldots,\gamma_n)$ o� $\gamma_i=\max(\alpha_i,\beta_i)$. $X^\gamma$ est appel� plus petit multiple commun de $\LM(f)$ et $\LM(g)$ et not� $\PPCM(\LM(f),\LM(g)):=X^\gamma$.
\item Le $S$-polyn�me de $f$ et $g$ est le polyn�me : $S(f,g):=\frac{X^\gamma}{\LT(f)}f-\frac{X^\gamma}{\LT(g)} g$
\end{enumerate}
\end{Def}

\begin{Thm} Soit $I$ un id�al de $k[X_1,\ldots,X_n]$. Alors une base $G=\{g_1,\ldots,g_n\}$ de $I$ est une base de Gr�bner de $I$ si, et seulement si, pour tout couple $(i,j)$, $i \neq j$, $\overline{S(g_i,g_j)}^G=0$
\end{Thm}


\subsection{Algorithme de Buchberger}
Maintenant que l'on a une caract�risation simple � v�rifier d'une base de Gr�bner, on va l'utiliser afin de cr�er un algorithme permettant d'en construire une. On va pr�senter l'algorithme dit de Buchberger cr�� par celui-ci. Cet algorithme consiste, pour une ensemble $F$ de polyn�mes, � calculer tous les restes $\overline{S(g_i,g_j)}^F$ et les ajouter � l'ensemble s'ils sont non nuls et recommencer avec le nouvel ensemble obtenu jusqu'� ce que cela converge. 
\begin{Thm}
Soit $I:=\left\langle f_1,\ldots,f_s \right\rangle$ un id�al non nul de $k[X_1,\ldots,X_n]$. Alors on peut construire, en temps fini, une base de Gr�bner de $I$ gr�ce � l'algorithme suivant.


\begin{algorithm}

\caption{Algorithme de Buchberger}

\begin{algorithmic}

\REQUIRE $F=(f_1,\ldots,f_s)$

\ENSURE Une base de Gr�bner $G=(g_1,\ldots,g_t)$ de $I $, avec $F \subset G $

\STATE $G:=F $

\REPEAT

\STATE $G':=G $

\FOR{chaque paire $\{p,q\}\in G'^2$, $p \neq q $}

\STATE $S:=\overline{S (p,q)}^{G'} $

\IF{$S \neq 0$}

\STATE $G:=G \cup \{S\} $

\ENDIF
\ENDFOR
\UNTIL $G=G'$

\end{algorithmic}

\end{algorithm}
\end{Thm}
\begin{proof}
Tout d'abord, on peut remarquer que $G \subset I$ � chaque �tape de l'algorithme car pour tout polyn�me $p,q \in I$, $S(p,q)$ est de la forme $fp+gq$ et est donc dans $I$. De ce fait, $S:=\overline{S(p,q)}^G$ appartient � $I$ et donc $G \cup \{S\} \subset I$. \\
Montrons maintenant que l'algorithme termine, c'est-�-dire que l'on a obtenu un ensemble $G$ tel que pour tout $p,q \in G, \overline{S(p,q)}^G=0$ c'est-�-dire une base de Gr�bner. \\
Consid�rons pour cela la suite $(G_j)_{j \in \N}$ d�finie comme la suite d'ensemble obtenu gr�ce � cet algorithme.\\
Comme pour tout $j \in \N$, $G_j \subset G_{j+1}$ alors $\left\langle \LT(G_j)\right\rangle\subset\left\langle \LT(G_{j+1})\right\rangle$. De plus, si $G_j \neq G_{j+1}$ alors $\left\langle \LT(G_j)\right\rangle\neq\left\langle \LT(G_{j+1})\right\rangle$. En effet, si $r \in G_{j+1} \setminus G_j$ alors, par d�finition du reste, $\LT(r)$ n'est pas divisible par aucun des �l�ments de $G_j$ et donc n'appartient pas � $\LT(G_j)$.
De plus, comme $k[X_1,\ldots,X_n]$est noeth�rien alors la suite $(\left\langle \LT(G_j)\right\rangle)$ croissante est stationnaire et donc la suite $(G_j)$ est stationnaire, ce qui montre que l'algorithme termine.
\end{proof}




La base de Gr�bner obtenue par cet algorithme contient trop d'�l�ments que n�cessaire comme l'illustre cet exemple : 
\begin{Ex}
Lorsqu'on calcule une base de Gr�bner pour l'id�al $\left\langle X^2 + Y^2 + Z^2 - 4, Y^2 + 2Z^2 - 5,XY - 1\right\rangle$, on obtient la famille $G=\{X^2 + Y^2 + Z^2 - 4, Y^2 + 2Z^2 - 5, XY - 1, X - YZ^2 +Y, 2XZ^2 - 5X + Y, -Z^4 + 7/2Z^2 - 3, 2YZ^6 - 12YZ^4+ 47/2YZ^2 - 15Y, -2Z^4 + 7Z^2 - 6, -YZ^4 + 7/2YZ^2 -3Y\}$. On peut remarquer que $X^2 + Y^2 + Z^2 - 4=Y^2 + 2Z^2 - 5+(Z^2 - 1)(XY - 1)+X(X - YZ^2 +Y)$ et que $G \setminus \{ X^2+Y^2+Z^2-4\}$ est toujours une base de Gr�bner. 
\end{Ex}
Le lemme suivant g�n�ralise cette remarque :
\begin{Lemme}
Soit $G $ une base de Gr�bner d'un id�al $I $ de $k[X_1,\ldots,X_n] $ et  $P\in G$ tel que $LT (P) \in \left\langle LT(G \setminus \{P\}) \right\rangle$. Alors $G \setminus \{P\} $ est une base de Gr�bner de $I $.
\end{Lemme}
\begin{proof}
Comme $G $ est une base de Gr�bner de $I $ alors $\left\langle LT(G)\right\rangle=\left\langle LT(I) \right\rangle$. Si $LT (P) \in \left\langle LT(G \setminus \{P\}) \right\rangle$ alors $\left\langle LT(G \setminus \{P\} )\right\rangle=\left\langle LT(G )\right\rangle=\left\langle LT(I) \right\rangle$, d'o� $G \setminus \{P\} $ est une base de Gr�bner de $I $.
\end{proof}
En it�rant ce processus, on obtient une base de Gr�bner minimale :
\begin{Def}
Une base de Gr�bner minimale $G$ d'un id�al $I $ de $k [X_1,\ldots,X_n] $ est une base de Gr�bner de $I $ telle que :
\begin{enumerate}
	\item  $\forall P \in G,  \LC(P)=1$

 \item $\forall P \in G, \LT(P) \notin \left\langle \LT(G \setminus \{P\}) \right\rangle$
\end{enumerate}
\end{Def}
Cependant pour un id�al donn�, on n'a pas l'unicit� de la base de Gr�bner minimal, comme l'illustre cet exemple : 
\begin{Ex}
Soit $I:= \left\langle X^2, XY, Y^2-1/2 X\right\rangle$ alors les familles $G_a=\{X^2+aXY, XY, Y^2-1/2 X \}$ sont des bases de Gr�bner minimale de $I$ pour tout $a \in k$.\\
\end{Ex}
L'unicit� est perdue � cause du fait qu'il y a des mon�mes d'un �l�ment $f$ de la base de $G_a$ qui sont dans $G \setminus\{f\}$. La notion de base de Gr�bner r�duite permet de contourner ce probl�me.
\begin{Def}

Une base de Gr�bner r�duite $G$ d'un id�al $I $ de $k [X_1,\ldots,X_n] $ est une base de Gr�bner de $I $ telle que :
\begin{enumerate}
	\item $\forall P \in G,  \LC(P)=1$
\item  Pour tout $P \in G$, aucun mon�me de $P $ n'appartient �  $\left\langle \LT(G \setminus \{P\}) \right\rangle$.
\end{enumerate}
\end{Def}
On obtient ainsi l'unicit� voulue : 
\begin{prop}
Soit $I $ un id�al non nul de $k [X_1,\ldots,X_n] $. Alors, pour un ordre monomial fix�, $I $ a une unique base de Gr�bner r�duite.
\end{prop}

\begin{proof} 
Soit $I $ un id�al non nul de $k [X_1,\ldots,X_n] $. \\
Commen�ons par montrer l'existence d'une base de Gr�bner r�duite d'un id�al $I$. \\
Soit $G$ une base de Gr�bner minimale de $I$. $g \in G$ est dit r�duit pour $G$ si aucun mon�me de $g$ n'est dans $\left\langle \LT(G \setminus \{g\})\right\rangle$. \\
On peut remarquer que $g$ est aussi r�duite pour n'importe quel base de Gr�bner minimale $G'$ telle que $\left\langle \LT(G)\right\rangle=\left\langle \LT(G')\right\rangle$.\\
Pour $g \in G$, posons $g':=\overline{g}^{G \setminus \{g\}}$ et $G':=(G \setminus \{g\} \cup \{g'\}$. Montrons que $G'$ est une base de Gr�bner minimale pour $I$. \\
Comme $G$ est une base de Gr�bner minimale alors $\LT(g) \notin \left\langle \LT(G \setminus \{g\}) \right\rangle$ et donc lorsqu'on divise $g$ par $G\setminus \{g\}$, $\LT(g')$ est ajout� au reste. De ce fait, $\LT(g)=\LT(g')$ et donc $\left\langle \LT(G)\right\rangle=\left\langle \LT(G')\right\rangle$. Comme $g' \in I$ alors $G' \subset I'$ et donc $G$ est une base de Gr�bner de $I$. De plus, la minimalit� de $G'$ se d�duit de celle de $G$ et $g'$ est r�duit pour $G'$ par construction (gr�ce � la condition sur le reste de l'algorithme de division) \\
En it�rant ce proc�d� pour tous les �l�ments de $G$, on finit par obtenir une base de Gr�bner r�duite $G'$ car tous les $g$ sont r�duits pour $G'$ (gr�ce � la remarque faite au d�but de la preuve).\\
Montrons maintenant l'unicit�. \\
Soit $G,\widetilde{G}$ deux bases de Gr�bner r�duites de $I$. \\
Montrons tout d'abord que $\LT(G)=\LT(\widetilde{G})$. Comme $G$ et $G'$ sont des bases de Gr�bner de $I$ alors $\left\langle \LT(G)\right\rangle=\left\langle \LT(\widetilde{G})\right\rangle$.\\
Soit $g \in G$ alors $\LT(g) \in \left\langle \LT(G') \right\rangle$ et donc d'apr�s le lemme \label{lem_div} il existe $g' \in \widetilde{G}$ tel que $\LT(g')$ divise $\LT(g)$. De la m�me fa�on, comme $\LT(g') \in \left\langle \LT(G)\right\rangle$ alors il existe $g'' \in G$ tel que $\LT(g'')$ divise $\LT(g')$ (et donc $\LT(g)$. Par minimalit� de $G$ et $G'$, $\LT(g)=\LT(g')=\LT(g'')$ et donc $\LT(g) \in G'$. D'o� $\LT(G) \subset \LT(g')$. Par sym�trie de r�le, $\LT(G)=\LT(\widetilde{G})$. \\
Soit $g \in G$ alors il existe un �l�ment $\widetilde{g} $ tel que $\LT(g)=\LT(\widetilde{g})$. Pour montrer que $G=\widetilde{G}$, il suffit de montrer que $g=\widetilde{g}$. \\
Comme $G$ est une base de Gr�bner de $I$ et que $g-\widetilde{g}\in I$ alors $\overline{g-\widetilde{g}}^G=0$. Comme $G$ et $\widetilde{G}'$ sont r�duites, alors aucun mon�me de $g-\widetilde{g}$ n'est divisible par un �l�ment de $\LT(G)$ d'o� par l'algorithme de division, $\overline{g-\widetilde{g}}^G=g-\widetilde{g}$ et donc $g-\widetilde{g}=0$ 
\end{proof}
On en d�duit un crit�re d'�galit� entre id�aux : 
\begin{cor}
Soient $I,J$ deux id�aux non nuls de $k[X_1,\ldots,X_n]$. Soient $G,H$ deux bases de Gr�bner r�duites de respectivement $I, J$.Alors $G=H \Leftrightarrow I=J$.
\end{cor}
\begin{Ex}
Soit $f,g,h$ d�finie dans l'exemple \ref{ex}. Alors la base r�duite de $\left\langle f,g,h\right\rangle $ est : $\{X - YZ^2 + Y, Y^2 + 2Z^2 - 5, Z^4 - 7/2Z^2 + 3\}$
\end{Ex}
 
\section{Th�or�me d'extension et d'�limination}
\section{G�om�trie}
\section{Implicitation}



\nocite{*}
\bibliographystyle{plain}
\bibliography{biblio}
\end{document}
