\documentclass{article}
\usepackage[utf8]{inputenc}
\usepackage{amsmath,amssymb}

\title{graphe}
\author{Moi}
\begin{document}
\maketitle
Soit $G:=(A,S)$ un graphe et $p\in \mathbb{N}^*$. \\
Soit $n:=Card(A)$.\\
Associons à chaque sommet de $G$ la variable $x_i$ et à chaque couleur une racine $p$ème de l'unité \emph{i.e.} $\forall i \in [\![1,n]\!],x_i^p=1$. \\
On impose de plus que si $x_i$ et $x_j$ sont adjacents alors $x_i \neq x_j$.
Cela revient à dire que $\sum_{k=0}^{p-1} x_i^k x_j^{p-1-k}=0$. \\
En effet, $0=x_i^p-x_j^p=\underbrace{(x_i-x_j)}_{\neq 0}\sum_{k=0}^{p-1} x_i^k x_j^{p-1-k}$.

$G$ est coloriable avec $p$ couleurs si, et seulement si, \\
le système $\begin{cases} \forall i \in [\![1,n]\!],x_i^p=1 \\
\forall i,j \in [\![1,n]\!], x_i \text{ et } x_j \text{ sont adjacents }, \sum_{k=0}^{p-1} x_i^k x_j^{p-1-k}=0
\end{cases}$ a une solution
\end{document}
