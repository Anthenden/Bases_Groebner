\begin{algorithm}

\caption{Algorithme de Buchberger}

\begin{algorithmic}

\REQUIRE $F=(f_1,\ldots,f_s)$

\ENSURE Une base de Gr�bner $G=(g_1,\ldots,g_t)$ de $I $, avec $F \subset G $

\STATE $G:=F $

\REPEAT

\STATE $G':=G $

\FOR{chaque paire $\{p,q\}\in G'^2$, $p \neq q $}

\STATE $S:=\overline{S (p,q)}^{G'} $

\IF{$S \neq 0$}

\STATE $G:=G \cup \{S\} $

\ENDIF
\ENDFOR
\UNTIL $G=G'$

\end{algorithmic}

\end{algorithm}
\begin{Lemme}
Soit $G $ une base de Gr�bner d'un id�al $I $ de $k[X_1,\ldots,X_n] $ et  $P\in G$ tel que $LT (P) \in \left\langle LT(G \setminus \{P\}) \right\rangle$. Alors $G \setminus \{P\} $ est une base de Gr�bner de $I $.
\end{Lemme}
\begin{demo}
Comme $G $ est une base de Gr�bner de $I $ alors $\left\langle LT(G)\right\rangle=\left\langle LT(I) \right\rangle$. Si $LT (P) \in \left\langle LT(G \setminus \{P\}) \right\rangle$ alors $\left\langle LT(G \setminus \{P\} )\right\rangle=\left\langle LT(G )\right\rangle=\left\langle LT(I) \right\rangle$ d'o� $G \setminus \{P\} $ est une base de Gr�bner de $I $.
\end{demo}

\begin{Def}
Une base de Gr�bner minimale d'un id�al $I $ de $k [X_1,\ldots,X_n] $ est une base de Gr�bner de $I $ telle que :
\begin{enumerate}
	\item  $\forall P \in G,  LC(P)=1$

 \item $\forall P \in G, LT (P) \notin \left\langle LT(G \setminus \{P\}) \right\rangle$
\end{enumerate}
\end{Def}
\begin{Def}

Une base de Gr�bner r�duite d'un id�al $I $ de $k [X_1,\ldots,X_n] $ est une base de Gr�bner de $I $ telle que :

\begin{enumerate}
	\item $\forall P \in G,  LC(P)=1$
\item  Pour tout $P \in G$, aucun mon�me de $P $ n'appartient �  $\left\langle LT(G \setminus \{P\}) \right\rangle$.
\end{enumerate}
\end{Def}
\begin{prop}
Soit $I $ un id�al non nul de $k [X_1,\ldots,X_n] $. Alors, pour un ordre monomial fix�, $I $ a une unique base de Gr�bner r�duite.
\end{prop}
\begin{demo} A faire \end{demo}