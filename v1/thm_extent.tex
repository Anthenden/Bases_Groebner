Soit $S_1$ le syst\`eme $\begin{cases}x^2=y \\ x^2=z \end{cases}$ et son ensemble de solutions $Z(x^2-y,x^2-z)$. Notons I=\left\langle x²-y,x²-z \right\rangle et $I_1$ son premier idéal d'élimination.\\
On peut calculer une base de Gröbner de $I$ : $I=\left\langle x^2-z,y-z\right\rangle$ d'où $I_1=<y-z>$
D'où $Z(I_1)=\{(c,c) | c \in k\}$ \\
On peut remarquer que les termes dominants de $x^2-y$ et $x^2-z$ ne s'annulent pas. D'où, d'après le théorème d'extension, on peut étendre toutes les solutions partielles dans $\C$.\\
Si on travaille dans $\R$, on peut étendre la solution $(c,c)$ en une solution de $S_1$ si, et seulement si $c\geq 0$. 
Soit $S_2$ le syst\`eme $\begin{cases}xy=1 \\ xz=1 \end{cases}$ et $I=\left\langle xy-1,xz-1 \right\rangle$. \\
On peut calculer une base de Gröbner de $I$ : $I=\left\langle x^2-z,y-z\right\rangle$ d'où $I_1=<y-z>$
D'où $Z(I_1)=\{(c,c) | c \in \C\}$ \\
On peut étendre toutes les solutions partielles sauf la solution $(0,0)$ où les termes dominants de $xy-1$ et $xz-1$ en $x$ s'annulent


