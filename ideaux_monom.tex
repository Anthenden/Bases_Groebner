\begin{Def} Un id�al monomial est un id�al de $k[X_1,\ldots,X_n]$ tel qu'il existe une partie $A$ de $\mathbb{N}^n$ telle que $ I=\left\langle X^{\alpha} | \alpha \in A \right\rangle=\{\sum P_\alpha X^\alpha | P_\alpha \in k[X_1,\ldots,X_n]\}$.
\end{Def}

 \begin{Lemme}
Soit $I:=\left\langle X^{\alpha} | \alpha \in A \right\rangle$ un id�al monomial. Alors $X^beta \in I$ ssi il existe un $\alpha \in A$ tel que $X^\alpha $divise $X^\beta$.
\end{Lemme}

\begin{demo}
  $\Leftarrow $ Evident \\
$\Rightarrow $ Si $X^\beta \in I$ alors il existe une famille de polyn�mes $P_1,\ldots,P_s \in k[X_1,\ldots,X_n]$ et d'exposants $\alpha_1,\ldots,\alpha_s \in \mathbb{N}^n$ telle que $X^\beta = \sum_{i=1}^s P_i X^{\alpha_i}$. \\
On peut alors remarquer qu'en utilisant les expressions $P_i:=\sum p_{i,\alpha} X^\alpha$ alors $X^\beta$ est de la forme $\sum_{\gamma \in \Gamma} p_\gamma X^{\gamma}$ o� $\Gamma:=\{\gamma \in \mathbb{N}^n | \exists n \in \mathbb{N}^n,\exists i \in [\![1,s]\!],\gamma=\alpha_i+n\}$. \\
Et donc $X^\beta- \sum_{\gamma \in \Gamma} p_\gamma X^{\gamma}=0 \; (*)$ \\
Comme $k[X_1,\ldots,X_n]$ est un $k$-espace vectoriel dont $\mathscr{M}$ est une base, on d�duit de $(*)$ que $p_\gamma=\begin{cases}
 0 \text{ si } \gamma \neq \beta \\
 1 \text{ sinon }
\end{cases}$ 
(dans le cas contraire, on aurait une combinaison lin�aire (d'�l�ment d'une base) nulle � coefficients non nuls). On en d�duit que $\beta \in \Gamma$ et donc qu'il existe un $n \in \mathbb{N}^n$ et un  $i \in [\![1,s]\!]$, $\beta=a_i+n $ c'est-�-dire qu'il existe un $i \in [\![1,s]\!]$ tel que $X^{\alpha_i}$ divise $X^\beta$.
\end{demo}

\begin{Lemme} Soit $I$ un id�al monomial et $f \in k[X_1,\ldots,X_n]$.

Les propositions suivantes sont �quivalentes :
\begin{enumerate}
 \item $f \in I$.
 \item Tous les termes de $f$ sont dans $I$.
 \item $f$ est une $k$-combinaison lin�aire de mon�mes dans $I$.
\end{enumerate}
\end{Lemme}
\begin{demo} 
$(3) \Rightarrow (2) \Rightarrow (1)$ est �vident. \\
$(1) \Leftrightarrow (3)$ se montre comme le lemme pr�c�dent.
\end{demo}

\begin{cor}Deux id�aux monomiaux sont �gaux ssi ils contiennent les m�mes mon�mes.
\end{cor}

\begin{demo}
 $\Rightarrow $ Evident \\
 $\Leftarrow $ Soit $I,I'$ deux id�aux monomiaux tel que $I \cap \mathscr{M}=I' \cap \mathscr{M}$. \\
Si $f:=\sum p_\alpha X^\alpha$ alors d'apr�s le lemme pr�c�dent, pour tout $\alpha \in A$, le mon�me $X^\alpha \in I$ alors, par hypoth�se, $X^\alpha \in I' \cap \mathscr{M}$ d'o� $X^\alpha \in I'$ et en r�utilisant le lemme, $f \in I'$. \\
On en d�duit que $I \subset I'$ et donc par sym�trie de r�le de $I$ et $I'$, $I =I'$. \\
\end{demo}

\begin{Lemme} Soit $I:=\left \langle X^\alpha | \alpha \in A \right\rangle$ un id�al monomial et supposons qu'il ait une base finie $\left \langle X^{\beta_1},\ldots,X^{\beta_s} \right\rangle$.
Supposons aussi qu'il existe une famille $\alpha_1,\ldots,\alpha_s$ tel que pour tout $i \in [\![1,s]\!]$,$X^{\alpha_i}$ divise $X^{\beta_i}$$(*)$ alors $I=\left\langle X^{\alpha_1},\ldots,X^{\alpha_s} \right\rangle$.
\end{Lemme}
\begin{demo} D'apr�s $(*)$, on a : $\forall i \in [\![1,s]\!],X^{\beta_i} \in \left\langle X^{\alpha_1},\ldots,X^{\alpha_s}\right\rangle$. D'o�,comme on a,de plus,$X^{\alpha_1},\ldots,X^{\alpha_s} \in I$,\\
$I=\left\langle X^{\beta_1},\ldots,X^{\beta_s}\right\rangle \subset \left\langle X^{\alpha_1},\ldots,X^{\alpha_s}\right\rangle \subset I $.\\
On en d�duit que $I=\left\langle X^{\alpha_1},\ldots,X^{\alpha_s} \right\rangle$
\end{demo}

\begin{Thm}[Lemme de Dickson]
Un id�al monomial $I:=\left\langle X^\alpha | \alpha \in A \right\rangle$ peut �tre �crit sous la forme $I=\left\langle X^{\alpha_1},\ldots,X^{\alpha_s} \right\rangle$, o� $\alpha_1,\ldots,\alpha_s$. En particulier, $I$ admet une base finie.
\end{Thm}

\begin{demo}  A faire. 
\end{demo}