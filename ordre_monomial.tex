\subsection{G�n�ralit�s}
\begin{Def} 
Un ordre monomial est une relation d'ordre total $\geq$ de $\mathscr{M}$ telle que :

\begin{enumerate}
	\item  $\forall \alpha,\beta,\gamma \in \mathbb{N}^n, X^\alpha \geq X^\beta \Rightarrow X^{\alpha+\gamma} \geq X^{\beta +\gamma}$
 \item $\geq$ est un bon ordre
\end{enumerate}
On note $X^\alpha > X^\beta$ si $X^\alpha \geq X^\beta$ et $\alpha \neq \beta$ (compatible avec l'addition) et $X^\alpha \leq X^\beta $ si $X^\beta \geq X^\alpha $
\end{Def}

\begin{prop}
Soit $\geq$ un ordre monomial. $1$ est le plus petit �l�ment de $\mathscr{M}$ pour $\geq$.
\end{prop}

\begin{demo}
Comme $\geq$ est un bon ordre alors il existe un plus petit �l�ment que l'on notera $x^\alpha$ alors :
$X^\alpha \leq 1$ et donc $X^{2 \alpha} \leq X^\alpha$ (par la compatibilit� avec l'addition). \\
Or comme $X^\alpha$ est le petit �l�ment de $\mathscr{M}$alors $X^\alpha \leq X^{2\alpha}$. \\
Donc, par antisym�trie, $X^\alpha=X^{2\alpha}$ d'o� $\alpha=2\alpha$ et donc $\alpha=0$. \\
On en d�duit que $1=X^0$ est le plus petit �l�ment de $\mathscr{M}$.
\end{demo}
\begin{cor}

Soit $\geq$ un ordre monomial et $\alpha,\beta \in \mathbb{N}^n$. Si $X^\alpha | X^\beta$ alors $X^\alpha \leq X^\beta$.
\end{cor}

\begin{demo}
  Si $X^\alpha | X^\beta$ alors il existe $\gamma \in \mathbb{N}^n$ tel que : $X^\beta=X^\gamma X^\alpha$. Or comme $1 \leq X^\gamma$ alors par compatibilit� avec l'addition, $X^\alpha \leq X^{\alpha+\gamma}=X^\beta$.
\end{demo}

\begin{Def}
Soit $P:=\sum_{\alpha} p_{\alpha}X^\alpha$ et $\geq$ un ordre monomial. \\
\begin{enumerate}
	\item  Le mon�me dominant de $P$ est : $LM(P):=max\{X^alpha \in \mathscr{M} | a_\alpha \neq 0\}$
\item Le multidegr� de $f$ est l'�l�ment de $\mathbb{N}^n$, not�$multideg(P)$, tel que $x^{multideg(P)}=LM(P)$
 \item Le coefficient dominant de $P$ est $LC(P):=a_{multideg(P)}$
\item Le terme dominant de $P$ est $LT(P):=LC(P)\cdot LM(P)$
\end{enumerate}


\end{Def}
\subsection{Exemples d'ordres monomiaux}
\begin{Defprop}[Ordre lexicographique $\geq_{lex}$]
Soient $\alpha=(\alpha_1,\ldots,\alpha_n), \beta=(\beta_1,\ldots,\beta_n) \in \mathbb{N}^n$ alors
$X^\alpha \geq_{lex} X^\beta$ si, et seulement si, $\alpha=\beta$ ou le premier coefficient non nul en lisant par la gauche de $\alpha -\beta $ est positif.
\end{Defprop}
\begin{demo}
Montrons que $\geq$ est un ordre monomial. \\
$\alpha,\beta,\gamma$ d�signeront des �l�ments quelconques de $\mathbb{N}^n$ et si $\alpha \neq \beta$, $\ell(\alpha,\beta)$ d�signera la premi�re composante ,en partant de la gauche,non nulle de $\alpha-\beta$  i.e. $\ell(\alpha,\beta):=min\{r \in [\![1,n]\!] | a_r \neq b_r \}$. \\
Montrons tout d'abord que c'est bien une relation d'ordre. \\
\textbf{R�flexivit� :}\\
$X^\alpha \geq X^\alpha$ (c.f. premier cas)\\
\textbf{Antisym�trie :} \\
Supposons que $X^\alpha \geq X^\beta $(i) et $X^\beta \geq X^\alpha$(ii). \\
Supposons, par l'absurde, que $X^\alpha \neq X^\beta$. \\
On a avec (i) que $\alpha_{\ell(\alpha,\beta)} >  \beta_{\ell(\alpha,\beta)}$ et avec (ii) que $\alpha_{\ell(\alpha,\beta)} <  \beta_{\ell(\alpha,\beta)}$. D'o� une contradiction. \\
On a donc $X^\alpha = X^\beta $. \\
\textbf{Transitivit� :} \\
Supposons que $X^\alpha \geq X^\beta$(i) et $X^\beta \geq X^\gamma$(ii).\\
Si $\alpha=\beta$, $\alpha=\gamma$ ou $\alpha=\beta$ alors l'in�galit� $X^\alpha \geq X^\gamma$ est �vidente.\\
Sinon, posons $\ell:=min\{\ell(\alpha,\beta),\ell(\beta,\gamma)\}$. \\
On a avec (i) et (ii),que : $\alpha_\ell < \beta_\ell \leq \gamma_\ell$ ou $\alpha_\ell \leq \beta_\ell <\gamma_\ell$ et pour tout $k < \ell$, $\alpha_\ell = \beta_\ell = \gamma_\ell$. \\
On a donc $X^\alpha \geq X^\gamma$. \\
\textbf{Montrons que $\geq_{lex}$ est compatible avec l'addition.}\\
Si $\alpha=\beta$ alors $X^{\alpha+\gamma}=X^{\beta+\gamma}$ et donc $X^{\alpha+\gamma}\geq_{lex}X^{\beta+\gamma}$\\
Sinon,  comme $(\alpha+\gamma)-(\beta+\gamma)=\alpha-\beta$ alors $\ell(\alpha,\beta)=\ell(\alpha+\gamma,\beta+\gamma)$ et donc si $X^{\alpha} \geq_{lex} X^\beta$ alors $X^{\alpha+\gamma} \geq_{lex} X^{\beta+\gamma}$.\\
\textbf{Montrons maintenant que $\geq_{lex}$ est un bon ordre}, par l'absurde.\\
Supposons donc que  $\geq_{lex}$ n'est pas un bon ordre et donc qu'il existe une suite $u:=\left(X^{(a_{1,i}\ldots,a_{n,i})}\right)_{i \in \mathbb{N}}$ strictement d�croissante.\\
On en d�duit que la suite $u_1:=(a_{1,i})_{i \in \mathbb{N}}$ est d�croissante (sinon $u$ ne serait pas d�croissante) et est donc stationnaire car $\mathbb{N}$ est bien ordonn�.\\
Alors il existe $N_1 \in \mathbb{N}$ tel que $\forall p \geq N,u_{1,p}=u_{1,N_1}$. \\
Consid�rons maintenant la suite $u_2:=(a_{2,i})_{i \geq N_1}$. Elle est d�croissante et donc stationnaire ... \\
On construit ainsi une suite $(N_i)_{i \geq 1}$ tel que $\forall n \geq N_i, u_{i,n} \geq u_{i,N_i}$.\\
On en d�duit que $\forall p \geq N_n,\forall i \in [\![1,n]\!], u_{i,p}=u_{i,N_n}$ ou encore $\forall p \geq N_n,, X^{u_{1,p},\ldots,u_{n,p}}=X^{u_{1,N_n},\ldots,u_{n,N_n}}$, ce qui est contradictoire avec la d�croissance de $u$.
\end{demo}
\begin{Defprop}[Ordre lexicographique gradu� $\geq_{grlex}$]
Soient $\alpha=(\alpha_1,\ldots,\alpha_n), \beta=(\beta_1,\ldots,\beta_n) \in \mathbb{N}^n$ alors
$X^\alpha \geq_{grlex} X^\beta$ si, et seulement si, $|\alpha|>|\beta|$ ou ($|\alpha|=|\beta|$ et $\alpha\geq_{lex} \beta$).
\end{Defprop}
\begin{Defprop}[Ordre lexicographique gradu� renvers� $\geq_{grevlex}$]
Soient $\alpha=(\alpha_1,\ldots,\alpha_n), \beta=(\beta_1,\ldots,\beta_n) \in \mathbb{N}^n$ alors
$X^\alpha \geq_{grevlex} X^\beta$ si, et seulement si, $|\alpha|>|\beta|$ ou ($|\alpha|=|\beta|$ et   le premier coefficient non nul en lisant par la droite de $\beta - \alpha $ est positif).
\end{Defprop}
\begin{Ex}
Ordre lexicographique : \\
$ X_1 >_{lex} X_2 >_{lex}\ldots >_{lex} X_n $\\
Pour $n=3$,
$X^2Y^2Z^4 >_{lex} X^1Y^4Z^{42} $ \\
$X^3Y^2Z^4 >_{lex} X^3Y^2Z^3$\\
Ordre lexicographique gradu�e : \\
$ X_1 >_{grlex} X_2 >_{grlex}\ldots >_{lex} X_n $ \\
Pour $n=3$, \\
$XY^4Z^8 >_{grlex} X^7Y^2Z^3$\\
$X^4Y^7Z >_{grlex} X^3Y^3Z^6$ \\
Ordre lexicographique gradu�e renvers�e : \\
$ X_1 >_{grevlex} X_2 >_{grevlex}\ldots >_{lex} X_n $ \\
Pour $n=3$, \\
$X^5Y^3Z^2 >_{grevlex} X^3Y^2Z^4$ \\
$X^4Y^3Z^2 >_{grevlex} X^2Y^5Z^2$ \\
\end{Ex}
