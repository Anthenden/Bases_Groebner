\subsection{Polynômes irréductibles et factorisation}
\begin{Def}
Un polynôme $P \in k[X_1,\ldots,X_n]$ est irréductible sur $k$ si $P$ est non constant et qu'il n'est pas le produit de deux polynômes non constants de $k[X_1,\ldots,X_n]$
\end{Def}
\begin{Prop}
Tout polynôme non constant de $k[X_1,\ldots,X_n]$ peut s'écrire comme produit de polynômes irréductible sur $k$.
\begin{Thm}
Soit $P \in k[X_1,\ldots,X_n]$ irréductible sur $k$ et supposons que $P$ divise le produit $QR$, avec $Q,R \in k[X_1,\ldots,X_n]$. Alors $P$ divise $Q$ ou $R$.
\end{Thm}
\begin{Thm} 
Tout polynôme non constant $f\in k[X_1,\ldots,X_n]$ peut s'écrire comme un produit $f=f_1 \ldots f_r$ d'irréductible sur $k$. De plus, $f=g_1\ldots g_s$ est une autre factorisation en irréductible sur $k$, alors $r=s$ et les $g_i$ peuvent être permutés de tel sorte que pour tout $i$,  $g_i$ soit un multiple de $f_i$ .
\end{Thm}
\subsection{Résultants}
cf Tout en un L2
\end{Def}