\documentclass[a4page,10pt]{article}
\usepackage[Latin1]{inputenc}
\usepackage[francais]{babel}
\usepackage{amsmath,amssymb,amsthm}
\usepackage{textcomp}
\usepackage{mathrsfs}
\usepackage{ algorithm , algorithmic }
\usepackage[all]{xy}
\usepackage{hyperref}

%%% francisation des algorithmes

\makeatletter

\renewcommand*{\ALG@name}{Algorithme}

\makeatother

\renewcommand{\algorithmicrequire}{\textbf{\textsc {Entr�es  :  } } }
\renewcommand{\algorithmicensure}{\textbf{\textsc { Sortie  :  } } }
\renewcommand{\algorithmicwhile}{\textbf{Tant que}}
\renewcommand{\algorithmicdo}{\textbf{faire }}
\renewcommand{\algorithmicif}{\textbf{Si}}
\renewcommand{\algorithmicelse}{\textbf{Sinon}}
\renewcommand{\algorithmicthen}{\textbf{alors }}
\renewcommand{\algorithmicend}{\textbf{fin}}
\renewcommand{\algorithmicfor}{\textbf{Pour}}
\renewcommand{\algorithmicuntil}{\textbf{Jusqu'�}}
\renewcommand{\algorithmicrepeat}{\textbf{R�p�ter}}

\newcommand{\F}{\mathbb{F}}
\newcommand{\K}{\mathbb{K}}
\newcommand{\Z}{\mathbb{Z}}
\newcommand{\N}{\mathbb{N}}
\newcommand{\B}{\mathcal{B}}
\newcommand{\GL}{\mathcal{G}\mathcal{L}}
\newcommand{\SL}{\mathcal{S}\mathcal{L}}
\renewcommand{\L}{\mathcal{L}}
\newcommand{\R}{\mathbb{R}}
\newcommand{\C}{\mathbb{C}}
\newcommand{\Ccal}{\mathcal{C}}
\newcommand{\Rcal}{\mathcal{R}}
\newcommand{\Ecal}{\mathcal{E}}
\newcommand{\Fcal}{\mathcal{F}}
\newcommand{\Mcal}{\mathcal{M}}
\newcommand{\Acal}{\mathcal{A}}
\newcommand{\Lcal}{\mathcal{L}}
\newcommand{\Rscr}{\mathscr{R}}
\newcommand{\Cscr}{\mathscr{C}}
\newcommand{\Dscr}{\mathscr{D}}
\newcommand{\Ebar}{\overline{E}}
\newcommand{\xbar}{\overline{x}}
\newcommand{\ybar}{\overline{y}}
\newcommand{\fbar}{\overline{f}}
\newcommand\bigzero{\makebox(0,0){\text{\huge0}}}

\DeclareMathOperator{\multideg}{\mathrm{multideg}}
\DeclareMathOperator{\LM}{\mathrm{LM}}
\DeclareMathOperator{\LT}{\mathrm{LT}}
\DeclareMathOperator{\LC}{\mathrm{LC}}
\DeclareMathOperator{\PPCM}{\mathrm{PPCM}}
\DeclareMathOperator{\PGCD}{\mathrm{PGCD}}
\DeclareMathOperator{\Syl}{\mathrm{Syl}}
\DeclareMathOperator{\Res}{\mathrm{Res}}
\DeclareMathOperator{\Com}{\mathrm{Com}}
\headheight=0mm
\topmargin=-10mm
\oddsidemargin=-1cm
\evensidemargin=-1cm
\textwidth=18cm
\textheight=22cm
\parindent=0mm
\begin{document}
\newtheorem{Thm}{Th�or�me}[section]
\newtheorem{prop}[Thm]{Propri�t�}
\newtheorem{Lemme}[Thm]{Lemme}
\newtheorem{cor}[Thm]{Corollaire}

\theoremstyle{definition}

\newtheorem{Ex}[Thm]{Exemple}
\newtheorem{Def}[Thm]{D�finition}
\newtheorem{Not}[Thm]{Notation}
\theoremstyle{remark}
\newtheorem{Rq}[Thm]{Remarque}

Soient les deux courbes alg�briques planes $\Cscr=\{f=0\},\Dscr=\{g=0\} \subset \C^2$ o� $f=\sum_{i=0}^m f_i(X)Y^i,g=\sum_{i=0}^n g_i(X)Y^i \in \C[X,Y]$.\\
Pour trouver les points d'intersection de $X$ et de $Y$, nous allons tout d'abord trouver les �l�ments du projet� $\pi_x(\Cscr \cap \Dscr)$ de leur intersection  sur un des axes (celui des $x$ ici) gr�ce � la caract�risation suivante : \\
Soit $a \in \C$ tel que $f_m(a)$ et $g_n(a)$ ne sont pas simultan�ment nuls. \\
$a \in \pi_x(\Cscr \cap \Dscr) \Leftrightarrow \exists b \in \C, (a,b) \in X \cap Y \Leftrightarrow \exists b \in \C, f(a,b)=g(a,b)=0$\\
$\Leftrightarrow  \PGCD(f(a,Y),g(a,Y))\neq 1$ car le PGCD de ces deux polyn�mes est divisible par $(X-b)$ gr�ce � l'�quivalence pr�c�dente \\
$\Leftrightarrow  \Res(f(a,Y),g(a,Y))=0$   gr�ce aux propri�t�s du r�sultant. \\
$\Leftrightarrow  \Res_Y(f,g)(a)=0$    car $f_m(a)$ et $g_n(a)$ ne sont pas simultan�ment nuls.  \\
Autrement dit, $\pi_x(\Cscr \cap \Dscr) \subset Z(\Res_Y(f,g))$.\\
De plus, pour pouvoir (�ventuellement) compl�ter une solution partielle $a$ de $\pi_x(\Cscr \cap \Dscr)$ en une solution de $X \cap Y$, il suffit de calculer les racines $\{x_1,\ldots,x_n\}$ de $\PGCD(f(a,Y),g(a,Y))$ et ensuite de v�rifier si les couples $(a,x_i)$ appartiennent � $\Cscr \cap \Dscr$.
\begin{Ex}
Supposons $f=(Y^2+6)(X^2-1)-Y(X^2+1), g=(Y^2+6)(X^2-1)-Y(X^2+1)$. \\
On commence par calculer le r�sultant de ces polyn�mes (en Y) :
$\Res_Y(f,g)=2(X-2)^2(X-3)^2(X^2-X+4)$. 
Ce polyn�me a 4 racines $2,3,\frac{1+i\sqrt{15}}{2},\frac{1-i\sqrt{15}}{2}$. \\
On essaie ensuite de compl�ter ces solutions partielles en calculant et en trouvant les racines de $\PGCD(f(\cdot,Y),g(\cdot,Y))$ : \\
$\PGCD(f(2,Y),g(2,Y))=(Y-2)(Y-3)$,
$\PGCD(f(3,Y),g(3,Y))=(Y-2)(Y-3)$\\
$\PGCD(f(\frac{1-i\sqrt{15}}{2},Y),g(\frac{1-i\sqrt{15}}{2},Y))=Y-\frac{1+i\sqrt{15}}{2}$,
$\PGCD(f(\frac{1+\sqrt{15}i}{2},Y),g(\frac{1+\sqrt{15}i}{2},Y))=Y-\frac{1-i\sqrt{15}}{2}$\\
On obtient donc les 6 points d'intersection de $\Cscr \cap \Dscr$ : $(2,2),(2,3),(3,2),(3,3),\left(\frac{1+i\sqrt{15}}{2},\frac{1-i\sqrt{15}}{2}\right),\left(\frac{1+i\sqrt{15}}{2},\frac{1-i\sqrt{15}}{2}\right)$
\end{Ex}



\end{document}