$\begin{cases}
x^2-yz=4 \\
y^2-xz=5 \\
z^2-xz=6
\end{cases}$
Base de Gr�bner r�duite : [Z^2 - 256/45, X + 7/8*Z, Y - 1/16*Z]
I2=Z^2 - 256/45
I1=Y - 1/16*Z,Z^2 - 256/45



\begin{Thm}
Soit $F:k^m \setminus Z(g_1 \ldots g_n) \to k^n$ une fonction d\'etermin\'ee par la param\'etrisation rationnelle $(\ddagger)$.\\ Soit $I$ l'id\'eal $\left\langle g_1X_1-f_1,\ldots,g_nX_n-f_n,1-g_1 \ldots g_n Y \right\rangle \subset k[Y,T_1,\ldots,T_m,X_1,\ldots,X_n]$ et $I_{m+1}$ son $m+1$ \`eme id\'eal d'\'elimination. Alors $Z(I_{m+1})$ est la plus petite vari�t� de $k^n$ contenant $F(k^m \setminus W)$.
\end{Thm}

$\begin{cases}
x=\frac{u^2}{v} \\
y=\frac{v^2}{u}\\
z=u
\end{cases}$

$I=\left\langle VX-U^2,UY-V^2,Z-U\right\rangle$
Base de Gr�bner : [U - Z, V^2 - Y*Z, V*X - Z^2, V*Z^2 - X*Y*Z, X^2*Y*Z - Z^4]
I_2=X^2YZ - Z^4
Z(I_2)=Z(X^2Y - Z^3) \cup Z(Z)
Donc $Z(I_2)$ n'est la plus petite vari�t� contenant $F((k^*)^2)$.

Folium de Descartes
$\begin{cases} 
x=\frac{3t}{1+t^3} \\
y=\frac{3t^2}{1+t^3}\\

\end{cases}$

Base de Gr�bner [T^2*Y - 3*T + X, T*X - Y, T*Y^2 + X^2 - 3*Y, X^3 - 3*X*Y + Y^3]

X^3 - 3*X*Y + Y^3

$S=Z(X^3 - 3*X*Y + Y^3)=:V$ 
En consid�rant les points d'intersection de $V$ et des droites $\Dscr_t$ d'�quation $y=tx$, $t \in \R^*$ : $(0,0)$, $(\frac{3t}{1+t^3},\frac{3t^2}{1+t^3})$, on montre l'�galit� entre la vari�t� $V$ et $S$.
