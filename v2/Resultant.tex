Soient $A $ un anneau commutatif int�gre et $K $ son corps de fractions.\\
Soient $f\in A [X] $ et $g \in A [X] $. Posons $p:=\deg f $ et $q:=\deg g $.\\
Soit $\varphi_{f,g} : K_{q-1}[X]\times K_{p-1}[X] \to K_{p+q-1}[X] $ l'application lin�aire d�finie par :\\ $\forall (u,v) \in K_{q-1}[X]\times K_{p-1}[X], \varphi(u,v)=fu+gv $.\\
On appelle matrice de Sylvester la matrice associ�e � l'application $\varphi_{f,g} $ dans la base $\B=((X^i,X^j) | 1 \leq i \leq q-1 ; 1 \leq j \leq p-1) $ que l'on notera $\Syl(f,g)$.\\
On appelle r�sultant de $f $ et $g $ le d�terminant de cette matrice et on le note $\Res(f,g)$.

\begin{Lemme}

$\Res(f,g) \in A$

\end{Lemme}

\begin{proof}

Le determinant est polynomial en les coordonn�es de la matrice et les composantes de $\Syl(f, g) $ sont dans   $A$.

\end{proof}

\begin{prop}

Supposons que $A=k [Y] $ avec $k $ un corps. Alors :

$\Res (f, g)\in\left\langle f,g \right \rangle \cap k[Y] $

\end{prop}

\begin{proof}

 $\Res (f,g) \in k [Y] $ gr�ce au lemme pr�c�dent. \\
Montrons maintenant que $\Res (f,g) \in \left\langle f,g \right\rangle $.\\
Notons $M $ la matrice de Sylvester de $f $ et $g $. \\
Alors $M {}^t \Com(M)=\det (M)I $ $(*) $ o� $\Com(M)$ est la matrice des cofacteurs. \\
Comme les cofacteurs de $M $ sont dans $A $ (car polynomial en les coordonn�es) alors $\Com (M) $ est dans $\Mcal_{p+q-1}(A) $. \\
Donc en posant $M=(C_1,\ldots,C_{p+q-1})$ alors d'apr�s $(*) $, on a : \\
$MC_1=\begin{pmatrix} \det (A) \\ 0 \\ \vdots\\ 0 \end{pmatrix} $. \\
En notant par $c $ l'�l�ment de $ A_{q-1}[X]\times A_{p-1}[X] $ de colonne de coordonn�es $C_1$ dans la base $\B  $ d�finie plus haut, on a : $\varphi_{f,g}(c)=det (A) $, ce qui permet de conclure.
\end{proof}