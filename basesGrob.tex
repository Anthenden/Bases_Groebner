\documentclass[a4paper,12pt]{article}
\usepackage[Latin1]{inputenc}
\usepackage[francais]{babel}
\usepackage{amsmath,amssymb}
\usepackage{textcomp}
\usepackage{mathrsfs}
\usepackage{ algorithm , algorithmic }
\usepackage[all]{xy}
%%% francisation des algorithmes

\makeatletter

\renewcommand*{\ALG@name}{Algorithme}

\makeatother

\renewcommand{\algorithmicrequire}{\textbf{\textsc {Entr�es  :  } } }
\renewcommand{\algorithmicensure}{\textbf{\textsc { Sortie  :  } } }
\renewcommand{\algorithmicwhile}{\textbf{Tant que}}
\renewcommand{\algorithmicdo}{\textbf{faire }}
\renewcommand{\algorithmicif}{\textbf{Si}}
\renewcommand{\algorithmicelse}{\textbf{Sinon}}
\renewcommand{\algorithmicthen}{\textbf{alors }}
\renewcommand{\algorithmicend}{\textbf{fin}}
\renewcommand{\algorithmicfor}{\textbf{Pour}}
\renewcommand{\algorithmicuntil}{\textbf{Jusqu'�}}

\newcommand{\F}{\mathbb{F}}
\newcommand{\K}{\mathbb{K}}
\newcommand{\Z}{\mathbb{Z}}
\newcommand{\N}{\mathbb{N}}
\newcommand{\B}{\mathcal{B}}
\newcommand{\GL}{\mathcal{G}\mathcal{L}}
\newcommand{\SL}{\mathcal{S}\mathcal{L}}
\renewcommand{\L}{\mathcal{L}}
\newcommand{\R}{\mathbb{R}}
\newcommand{\C}{\mathbb{C}}
\newcommand{\Ccal}{\mathcal{C}}
\newcommand{\Rcal}{\mathcal{R}}
\newcommand{\Ecal}{\mathcal{E}}
\newcommand{\Fcal}{\mathcal{F}}
\newcommand{\Mcal}{\mathcal{M}}
\newcommand{\Acal}{\mathcal{A}}
\newcommand{\Lcal}{\mathcal{L}}
\newcommand{\Rscr}{\mathscr{R}}
\newcommand{\Ebar}{\overline{E}}
\newcommand{\xbar}{\overline{x}}
\newcommand{\ybar}{\overline{y}}
\newcommand{\fbar}{\overline{f}}

\begin{document}
\title{Bases de Gr�bner}
\author{Antoine BOIVIN}
\maketitle
\tableofcontents
\newtheorem{Thm}{Th�or�me}[section]
\newtheorem{prop}[Thm]{Propri�t�}
\newtheorem{demo}[Thm]{D�monstration}
\newtheorem{Lemme}[Thm]{Lemme}
\newtheorem{cor}[Thm]{Corollaire}
\newtheorem{Rq}[Thm]{Remarque}
\newtheorem{Ex}[Thm]{Exemple}
\newtheorem{Def}[Thm]{D�finition}
\newtheorem{Not}[Thm]{Notation}
\newtheorem{Defprop}[Thm]{D�finition et propri�t�}
\newtheorem{ThmDef}[Thm]{Th�or�me et d�finition}

\section{Ordre monomial}
\subsection{G�n�ralit�s}
\begin{Def} 
Un ordre monomial est une relation d'ordre total $\geq$ de $\mathscr{M}$ telle que :

\begin{enumerate}
	\item  $\forall \alpha,\beta,\gamma \in \mathbb{N}^n, X^\alpha \geq X^\beta \Rightarrow X^{\alpha+\gamma} \geq X^{\beta +\gamma}$(compatibilit� avec le produit)
 \item $\geq$ est un bon ordre
\end{enumerate}
On note $X^\alpha > X^\beta$ si $X^\alpha \geq X^\beta$ et $\alpha \neq \beta$  et $X^\alpha \leq X^\beta $ si $X^\beta \geq X^\alpha $
\end{Def}

\begin{prop}
Soit $\geq$ un ordre monomial. $1$ est le plus petit �l�ment de $\mathscr{M}$ pour $\geq$.
\end{prop}

\begin{proof}

Comme $\geq$ est un bon ordre alors il existe un plus petit �l�ment que l'on notera $X^\alpha$ alors :
$X^\alpha \leq 1$ et donc $X^{2 \alpha} \leq X^\alpha$ (par la compatibilit� avec le produit). \\
Or comme $X^\alpha$ est le petit �l�ment de $\mathscr{M}$alors $X^\alpha \leq X^{2\alpha}$. \\
Donc, par antisym�trie, $X^\alpha=X^{2\alpha}$ d'o� $\alpha=2\alpha$ et donc $\alpha=0$. \\
On en d�duit que $1=X^0$ est le plus petit �l�ment de $\mathscr{M}$.

\end{proof}

\begin{cor}

Soit $\geq$ un ordre monomial et $\alpha,\beta \in \mathbb{N}^n$. Si $X^\alpha$ divise $X^\beta$ alors $X^\alpha \leq X^\beta$.
\end{cor}

\begin{proof}
  Si $X^\alpha$ divise $X^\beta$ alors il existe $\gamma \in \mathbb{N}^n$ tel que : $X^\beta=X^\gamma X^\alpha$. Or, $1 \leq X^\gamma$, d'o�, par compatibilit� avec le produit, $X^\alpha \leq X^{\alpha+\gamma}=X^\beta$.
\end{proof}

\begin{Def}
Soit $P:=\sum_{\alpha} p_{\alpha}X^\alpha \in k[X_1,\ldots,X_n]$ et $\geq$ un ordre monomial. \\
\begin{enumerate}
	\item  Le mon�me dominant de $P$ est : $\LM(P):=\max\{X^\alpha \in \mathscr{M} | a_\alpha \neq 0\}$
\item Le multidegr� de $f$ est l'�l�ment de $\mathbb{N}^n$, not� $\multideg(P)$, tel que $x^{\multideg(P)}=LM(P)$
 \item Le coefficient dominant de $P$ est $\LC(P):=a_{\multideg(P)}$
\item Le terme dominant de $P$ est $\LT(P):=\LC(P)\cdot \LM(P)$
\end{enumerate}


\end{Def}
\subsection{Exemples d'ordres monomiaux}
\begin{Defprop}[Ordre lexicographique $\geq_{lex}$]
Soient $\alpha=(\alpha_1,\ldots,\alpha_n), \beta=(\beta_1,\ldots,\beta_n) \in \mathbb{N}^n$ alors
$X^\alpha \geq_{lex} X^\beta$ si, et seulement si, $\alpha=\beta$ ou le premier coefficient non nul en lisant par la gauche de $\alpha -\beta $ est positif.
\end{Defprop}
\begin{proof}
Montrons que $\geq_{lex}$ est un ordre monomial. \\
$\alpha,\beta,\gamma$ d�signeront des �l�ments quelconques de $\mathbb{N}^n$ de composantes respectives $\alpha_i,\beta_i,\gamma_i$ et si $\alpha \neq \beta$, $\ell(\alpha,\beta)$ d�signera la premi�re composante ,en partant de la gauche,non nulle de $\alpha-\beta$  i.e. $\ell(\alpha,\beta):=\min\{r \in [\![1,n]\!] | a_r \neq b_r \}$. \\
Montrons tout d'abord que c'est bien une relation d'ordre. \\
\textbf{R�flexivit� :}\\
$X^\alpha \geq X^\alpha$ (\emph{c.f.} premier cas)\\
\textbf{Antisym�trie :} \\
Supposons que $X^\alpha \geq X^\beta $(i) et $X^\beta \geq X^\alpha$(ii). \\
Supposons, par l'absurde, que $X^\alpha \neq X^\beta$. \\
On a avec (i) que $\alpha_{\ell(\alpha,\beta)} >  \beta_{\ell(\alpha,\beta)}$ et avec (ii) que $\alpha_{\ell(\alpha,\beta)} <  \beta_{\ell(\alpha,\beta)}$. D'o� une contradiction. \\
On a donc $X^\alpha = X^\beta $. \\
\textbf{Transitivit� :} \\
Supposons que $X^\alpha \geq X^\beta$(i) et $X^\beta \geq X^\gamma$(ii).\\
Si $\alpha=\beta$, $\alpha=\gamma$ ou $\alpha=\beta$ alors l'in�galit� $X^\alpha \geq X^\gamma$ est �vidente.\\
Sinon, posons $\ell:=\min\{\ell(\alpha,\beta),\ell(\beta,\gamma)\}$. \\
On a avec (i) et (ii),que : $\alpha_\ell > \beta_\ell \geq \gamma_\ell$ ou $\alpha_\ell \geq \beta_\ell >\gamma_\ell$ et pour tout $k < \ell$, $\alpha_\ell = \beta_\ell = \gamma_\ell$. \\
On a donc $X^\alpha \geq X^\gamma$. \\
\textbf{Montrons que $\geq_{lex}$ est compatible avec le produit.}\\
Si $\alpha=\beta$ alors $X^{\alpha+\gamma}=X^{\beta+\gamma}$ et donc $X^{\alpha+\gamma}\geq_{lex}X^{\beta+\gamma}$\\
Sinon,  comme $(\alpha+\gamma)-(\beta+\gamma)=\alpha-\beta$ alors $\ell(\alpha,\beta)=\ell(\alpha+\gamma,\beta+\gamma)$ et donc si $X^{\alpha} \geq_{lex} X^\beta$ alors $X^{\alpha+\gamma} \geq_{lex} X^{\beta+\gamma}$.\\
\textbf{Montrons maintenant que $\geq_{lex}$ est un bon ordre}, par l'absurde.\\
Supposons donc que  $\geq_{lex}$ n'est pas un bon ordre et donc qu'il existe une suite $u:=\left(X^{(a_{1,i}\ldots,a_{n,i})}\right)_{i \in \mathbb{N}}$ strictement d�croissante.\\
On en d�duit que la suite $u_1:=(a_{1,i})_{i \in \mathbb{N}}$ est d�croissante (sinon $u$ ne serait pas d�croissante) et est donc stationnaire car $\mathbb{N}$ est bien ordonn�.\\
Alors il existe $N_1 \in \mathbb{N}$ tel que $\forall p \geq N,u_{1,p}=u_{1,N_1}$. \\
Consid�rons maintenant la suite $u_2:=(a_{2,i})_{i \geq N_1}$. Elle est d�croissante et donc stationnaire ... \\
On construit ainsi une suite $(N_i)_{i \geq 1}$ tel que $\forall n \geq N_i, u_{i,n} \geq u_{i,N_i}$.\\
On en d�duit que $\forall p \geq N_n,\forall i \in [\![1,n]\!], u_{i,p}=u_{i,N_n}$ ou encore \\$\forall p \geq N_n, X^{u_{1,p},\ldots,u_{n,p}}=X^{u_{1,N_n},\ldots,u_{n,N_n}}$, ce qui est contradictoire avec la stricte d�croissance de $u$.
\end{proof}
\begin{Defprop}[Ordre lexicographique gradu� $\geq_{grlex}$]
Soient $\alpha=(\alpha_1,\ldots,\alpha_n), \beta=(\beta_1,\ldots,\beta_n) \in \mathbb{N}^n$ alors
$X^\alpha \geq_{grlex} X^\beta$ si, et seulement si, $|\alpha|>|\beta|$ ou ($|\alpha|=|\beta|$ et $\alpha\geq_{lex} \beta$).
\end{Defprop}
\begin{Defprop}[Ordre lexicographique gradu� renvers� $\geq_{grevlex}$]
Soient $\alpha=(\alpha_1,\ldots,\alpha_n), \beta=(\beta_1,\ldots,\beta_n) \in \mathbb{N}^n$ alors
$X^\alpha \geq_{grevlex} X^\beta$ si, et seulement si, $|\alpha|>|\beta|$ ou ($|\alpha|=|\beta|$ et le premier coefficient non nul en lisant par la droite de $\beta - \alpha $ est positif).
\end{Defprop}
\begin{Ex}
Ordre lexicographique : \\
$ X_1 >_{lex} X_2 >_{lex}\ldots >_{lex} X_n $\\
Pour $n=3$,\\
$X^2Y^2Z^4 >_{lex} X^1Y^4Z^{42} $ \\
$X^3Y^2Z^4 >_{lex} X^3Y^2Z^3$\\
Ordre lexicographique gradu�e : \\
$ X_1 >_{grlex} X_2 >_{grlex}\ldots >_{lex} X_n $ \\
Pour $n=3$, \\
$XY^4Z^8 >_{grlex} X^7Y^2Z^3$\\
$X^4Y^7Z >_{grlex} X^3Y^3Z^6$ \\
Ordre lexicographique gradu�e renvers�e : \\
$ X_1 >_{grevlex} X_2 >_{grevlex}\ldots >_{lex} X_n $ \\
Pour $n=3$, \\
$X^5Y^3Z^2 >_{grevlex} X^3Y^2Z^4$ \\
$X^4Y^3Z^2 >_{grevlex} X^2Y^5Z^2$ \\
\end{Ex}

\section{Algorithme de division}
  \begin{Lemme}Soit $\alpha,\alpha_1,\ldots,\alpha_n \in \mathbb{N}^n $ tel que : $X^\alpha >X^{\alpha_1}>\ldots >X^{\alpha_n}$. Soit $f,g \in k[X_1,\ldots,X_n]$ tels que $LT(f)=LT(g)$ alors $LM(f-g) <LT(f)=LT(g)$
	\end{Lemme}
	\begin{demo}
	Soit $f:=pX^\alpha+\sum p_{\alpha_i}X^{\alpha_i}$ et $g:=pX^\alpha+\sum q_{\alpha_i}X^{\alpha_i}$ alors $LM(f-g)=LM(\sum p_{\alpha_i}X^{\alpha_i}) \leq X^{\alpha_1} < X^\alpha=LM(f)=LM(g)$
\end{demo}
%****************************************************************
\begin{Thm}
\begin{algorithm}
\caption{Algorithme de division}
\begin{algorithmic}
\REQUIRE $f_1,\ldots,f_s,f$
\ENSURE $a_1,\ldots,a_s,r$
\STATE $a_1 :=0 ;\ldots ; a_s:=0 ; r:=0$
\STATE $p:=0$
\WHILE{$p \neq 0$}
\STATE $i:=1$
\STATE $divisionoccured:=false $
\WHILE{$i \leq s $ et $divisionoccured = false $}
\IF{$LT (f_i) | LT (p)$}
\STATE $a_i:= a_i+LT (p)/LT (f_i)$
\STATE $p:=p-(LT(p)/LT (f_i)) f_i$
\ELSE
\STATE $i:=i+1$
\ENDIF
\ENDWHILE
\IF{divisionoccured=false}
\STATE $r:=r+LT (p)$
\STATE $p:=p-LT (p) $
\ENDIF
\ENDWHILE
\end{algorithmic}
\end{algorithm}      %Probl�me typo 
\end{Thm}


%****************************************************************
\begin{demo}
Remarquons tout d'abord que lors de chaque it�ration de la boucle, une de ses deux instructions :
\begin{enumerate}
	\item Si $LT(f_i) | LT(p)$ alors on fait la division de $p$ par $f_i$
 \item Sinon on ajoute $LT(p)$ � $r$ (et on retire $LT(p)$ � $p$).
\end{enumerate}
Montrons d'abord que l'algorithme s'arr�te i.e. il existe une �tape o� $p=0$. \\
Pour cela, montrons que la suite des mon�mes dominants des diff�rentes valeurs $p$ est strictement d�croissante tant que $p \neq 0$. Si l'algorithme ne s'arr�tait pas, on aurait une suite infinie strictement croissante ce qui contredit le fait que $\geq$ est un bon ordre. \\

-Si on fait une division (par $f_j$) alors $p$ prend la valeur $p':=p-\frac{LT(p)}{LT(f_j)}f_j$. \\
-Si cette valeur est nulle alors l'algorithme s'arr�te sinon comme on a l'�galit� : $LT\left(\underbrace{\frac{LT(p)}{LT(f_j)}}_{\in k^*\mathscr{M}}f_j\right)=\frac{LT(p)}{LT(f_j)}LT(f_j)=LT(p)$.\\
On en d�duit donc,d'apr�s le lemme, que $LM(p') <LM(p)$. \\
-Sinon, $p$ prend la valeur $p-LT(p)$. Par le m�me argument que pr�c�demment, $LM(p-LT(p))<LT(p)$. \\
Ce qui permet de conclure. \\
Montrons maintenant qu'� chaque �tape que $f=\sum_{i=0}^s a_if_i +p+r$. \\
Initialisation de l'algorithme ("0�me it�ration") : Comme $a_1=\ldots=a_s=r=0$ et $p=f$ alors l'�galit� est v�rifi�e. \\
H�r�dit� : Soit $n \in \mathbb{N}$ et supposons qu'� la $n$�me it�ration de la boucle, $f=\sum_{i=0}^s a_if_i +p+r=\sum_{i=0,i \neq j}^s a_if_i +a_jf_j+p+r$ pour tout $j\in [\![1,n]\!]$  alors : \\
- si on fait une division ( $p$ avec $f_j$) alors : la nouvelle valeur $p'$ de $p$ est $p-\frac{LT(p)}{LT(f_j)}f_j$ et celle de $a_i$ est $a'_j=a_j+\frac{LT(p)}{LT(f_j)}$. et donc : \\
$\sum_{i=0,i \neq j}^s a_if_i +a'_jf_j+p'+r=\sum_{i=0,i \neq j}^s a_if_i +\left(a_j+\frac{LT(p)}{LT(f_j)}\right)f_j+p-\frac{LT(p)}{LT(f_j)}f_j+r$ \\
$=\sum_{i=0,i \neq j}^s a_if_i +a_jf_j+p+r=f$.
On obtient donc,lorsque $p=0$ (et on sait que cela arrivera), que  $f=\sum_{i=1}^s a_if_i+r$ et $r$ est, par d�finition dans l'algorithme, une somme d'�l�ments non divisibles par les $LT(f_i)$
\end{demo}
\section{Id�aux monomiaux}
\begin{Def} Un id�al monomial est un id�al de $k[X_1,\ldots,X_n]$ tel qu'il existe une partie $A$ de $\mathbb{N}^n$ telle que $ I=\left\langle X^{\alpha} | \alpha \in A \right\rangle=\{\sum P_\alpha X^\alpha | P_\alpha \in k[X_1,\ldots,X_n]\}$.
\end{Def}

 \begin{Lemme}
Soit $I:=\left\langle X^{\alpha} | \alpha \in A \right\rangle$ un id�al monomial. Alors $X^beta \in I$ ssi il existe un $\alpha \in A$ tel que $X^\alpha $divise $X^\beta$.
\end{Lemme}

\begin{proof}
  $\Leftarrow $ Evident \\
$\Rightarrow $ Si $X^\beta \in I$ alors il existe une famille de polyn�mes $P_1,\ldots,P_s \in k[X_1,\ldots,X_n]$ et d'exposants $\alpha_1,\ldots,\alpha_s \in \mathbb{N}^n$ telle que $X^\beta = \sum_{i=1}^s P_i X^{\alpha_i}$. \\
On peut alors remarquer qu'en utilisant les expressions $P_i:=\sum p_{i,\alpha} X^\alpha$ alors $X^\beta$ est de la forme $\sum_{\gamma \in \Gamma} p_\gamma X^{\gamma}$ o� $\Gamma:=\{\gamma \in \mathbb{N}^n | \exists n \in \mathbb{N}^n,\exists i \in [\![1,s]\!],\gamma=\alpha_i+n\}$. \\
Et donc $X^\beta- \sum_{\gamma \in \Gamma} p_\gamma X^{\gamma}=0 \; (*)$ \\
Comme $k[X_1,\ldots,X_n]$ est un $k$-espace vectoriel dont $\mathscr{M}$ est une base, on d�duit de $(*)$ que $p_\gamma=\begin{cases}
 0 \text{ si } \gamma \neq \beta \\
 1 \text{ sinon }
\end{cases}$ 
(dans le cas contraire, on aurait une combinaison lin�aire (d'�l�ment d'une base) nulle � coefficients non nuls). On en d�duit que $\beta \in \Gamma$ et donc qu'il existe un $n \in \mathbb{N}^n$ et un  $i \in [\![1,s]\!]$, $\beta=a_i+n $ c'est-�-dire qu'il existe un $i \in [\![1,s]\!]$ tel que $X^{\alpha_i}$ divise $X^\beta$.
\end{proof}

\begin{Lemme} Soit $I$ un id�al monomial et $f \in k[X_1,\ldots,X_n]$.

Les propositions suivantes sont �quivalentes :
\begin{enumerate}
 \item $f \in I$.
 \item Tous les termes de $f$ sont dans $I$.
 \item $f$ est une $k$-combinaison lin�aire de mon�mes dans $I$.
\end{enumerate}
\end{Lemme}
\begin{proof} 
$(3) \Rightarrow (2) \Rightarrow (1)$ est �vident. \\
$(1) \Leftrightarrow (3)$ se montre comme le lemme pr�c�dent.
\end{proof}

\begin{cor}Deux id�aux monomiaux sont �gaux ssi ils contiennent les m�mes mon�mes.
\end{cor}

\begin{proof}
 $\Rightarrow $ Evident \\
 $\Leftarrow $ Soit $I,I'$ deux id�aux monomiaux tel que $I \cap \mathscr{M}=I' \cap \mathscr{M}$. \\
Si $f:=\sum p_\alpha X^\alpha$ alors d'apr�s le lemme pr�c�dent, pour tout $\alpha \in A$, le mon�me $X^\alpha \in I$ alors, par hypoth�se, $X^\alpha \in I' \cap \mathscr{M}$ d'o� $X^\alpha \in I'$ et en r�utilisant le lemme, $f \in I'$. \\
On en d�duit que $I \subset I'$ et donc par sym�trie de r�le de $I$ et $I'$, $I =I'$. \\
\end{proof}

\begin{Lemme} Soit $I:=\left \langle X^\alpha | \alpha \in A \right\rangle$ un id�al monomial et supposons qu'il ait une base finie $\left \langle X^{\beta_1},\ldots,X^{\beta_s} \right\rangle$.
Supposons aussi qu'il existe une famille $\alpha_1,\ldots,\alpha_s$ tel que pour tout $i \in [\![1,s]\!]$,$X^{\alpha_i}$ divise $X^{\beta_i}$$(*)$ alors $I=\left\langle X^{\alpha_1},\ldots,X^{\alpha_s} \right\rangle$.
\end{Lemme}
\begin{proof} D'apr�s $(*)$, on a : $\forall i \in [\![1,s]\!],X^{\beta_i} \in \left\langle X^{\alpha_1},\ldots,X^{\alpha_s}\right\rangle$. D'o�,comme on a,de plus,$X^{\alpha_1},\ldots,X^{\alpha_s} \in I$,\\
$I=\left\langle X^{\beta_1},\ldots,X^{\beta_s}\right\rangle \subset \left\langle X^{\alpha_1},\ldots,X^{\alpha_s}\right\rangle \subset I $.\\
On en d�duit que $I=\left\langle X^{\alpha_1},\ldots,X^{\alpha_s} \right\rangle$
\end{proof}

\begin{Thm}[Lemme de Dickson]
Un id�al monomial $I:=\left\langle X^\alpha | \alpha \in A \right\rangle$ peut �tre �crit sous la forme $I=\left\langle X^{\alpha_1},\ldots,X^{\alpha_s} \right\rangle$, o� $\alpha_1,\ldots,\alpha_s$. En particulier, $I$ admet une base finie.
\end{Thm}

\begin{proof}  A faire. 
\end{proof}
\section{Bases de Gr�bner}
\subsection{G�n�ralit�s}
\begin{Not} Soit $I$ un id�al non r�duit � $\{0\}$ de $k[X_1,\ldots,X_n]$. \\
On note $LT(I)$ l'ensemble des termes dominants des �l�ments de $I$ i.e. $LT(I):=\{cX^\alpha| \exists f \in I, LT(f)=cX^\alpha\}$
\end{Not}
\begin{Lemme} Soient $A \subset k[X_1,\ldots,X_n]$ et $(p_i)_{i \in A}$ une suite d'�l�ments de $k^*$.Alors : \\
$\left\langle p_f f | f \in A\right\rangle= \left\langle A \right\rangle \;(*)$\\

\end{Lemme}

\begin{proof} On notera $I_1$ l'id�al � gauche de l'�galit� $(*)$ et $I_2$ celui de droite. \\
$\subset$ : Soit $P=\sum_{f \in A} \alpha_f (p_ff) \in I_1$ alors,par associativit� du produit, $P=\sum_{f \in A} (\alpha_f p_f)f \in I_2$ \\
$\supset$ : Soit $P=\sum_{f \in A}\alpha_f f \in I_2$ alors,par associativit� du produit,$P=\sum_{f \in A} \frac{\alpha_f}{p_f} (p_ff)\in I_1$ \\
\end{proof}
\begin{cor}
En particulier, $\left\langle LT(f) | f \in A \setminus \{0\}\right\rangle=\left\langle LM(f) |f \in A \setminus \{0\}\right\rangle$.
\end{cor}
\begin{proof}
Pour tout $f \in k[X_1,\ldots,X_n], f \neq 0$, $LT(f)=LC(f)LM(f)$.On conclut gr�ce au lemme
\end{proof}
\begin{prop}

Soit $I \subset k[X_1,\ldots,X_n]$ un id�al. Alors :
 \begin{enumerate}
	 \item $\left\langle LT(I)\right\rangle$ est un id�al monomial.
 \item Il existe $g_1,\ldots,g_s \in I$ tel que : $\left\langle LT(I)\right\rangle=\left\langle LT(g_1),\ldots,LT(g_s) \right\rangle$.

\end{enumerate}
\end{prop}
\begin{proof}

(1)D'apr�s le lemme pr�c�dent, on a $\left\langle LM(g) | g \in I \setminus \{0\} \right\rangle =\left\langle LT(g) | g \in I \setminus \{0\}\right\rangle=\left\langle LT(I)\right\rangle$ ce qui montre que $\left\langle LT(I)\right\rangle$ est un id�al monomial. \\
(2) Comme $\left\langle LT(I)\right\rangle$ est un id�al monomial engendr� par $LM(g)$ (avec $g \in I \setminus \{0\}$ ) alors, d'apr�s le lemme de Dickson, il existe $g_1,\ldots,g_s$ tel que : \\$LT(I)=\left\langle LM(g_1),\ldots,LM(g_s) \right\rangle$. On conclut en utilisant le lemme pr�c�dent : \\
$LT(I)= \left\langle LM(g_1),\ldots,LM(g_s) \right\rangle=\left\langle LT(g_1),\ldots,LT(g_s) \right\rangle$
\end{proof}

\begin{Thm}[de la base de Hilbert]  Tout id�al $I$ de $k[X_1,\ldots,X_n]$ admet une base finie.
\end{Thm}
\begin{proof}
 Si $I=\{0\}$ alors $I$ est engendr� par la famille finie $\{0\}$. \\
Sinon,  on a, d'apr�s la proposition pr�c�dente, l'existence de $f_1,\ldots,f_s \in I,$ tels que $\left\langle LT(I)\right\rangle=\left\langle LT(f_1),\ldots,LT(f_s) \right\rangle$. Montrons que $I=\left\langle f_1,\ldots,f_s \right\rangle$.\\
$\supset$ :  $f_1,\ldots,f_s \in I$ \\
$\subset$ : Soit $f \in I$ alors la division de $f$ par $f_1,\ldots,f_s$ s'�crit : $f=\sum_{i=1}^s \alpha_i f_i+r$ o� chaque terme de $r$ n'est pas divisible par des $LT(f_i)$. \\
Pour montrer l'inclusion, il nous faut montrer que $r=0$. \\
Supposons, par l'absurde, que $r \neq 0$.\\
On a $r=f-\sum_{i=1}^s \alpha_if_i \in I$ d'o� $LT(r) \in \left\langle LT(I) \right\rangle=\left\langle f_1,\ldots,f_s \right\rangle$. \\
Alors, d'apr�s le lemme \ref{},$LT(r)$ est divisible par un des $LT(f_i)$ ce qui est en contradiction avec la d�finition de $r$. \\
On en d�duit alors que $r=0$ et donc $f \in \left\langle f_1,\ldots,f_s \right\rangle$. \\
\end{proof}
\begin{Def}
Soit $\geq$ un ordre monomial. Un sous-ensemble $G=\{g_1,\ldots,g_s\}$ d'un id�al $I$ est une base de Gr�bner si $\left\langle LT(I) \right\rangle=\left\langle LT(g_1),\ldots,LT(g_s)\right\rangle$.
\end{Def}

\begin{cor}
 Soit $\geq$ un ordre monomial. Alors tout id�al de $k[X_1,\ldots,X_n]$ non r�duit � $\{0\}$ a une base de Gr�bner. De plus, tout base de Gr�bner est une base de $I$.
\end{cor}

\begin{proof}
 
 \begin{enumerate}
	 \item \emph{c.f.} prop \ref{}
	\item \emph{c.f.} d�monstration du th�or�me de la base de Hilbert.
 \end{enumerate}
\end{proof}
\subsection{Propri�t�s des bases de Gr�bner}
\begin{prop}
 Soit $G=\{g_1,\ldots,g_s\}$ une base de Gr�bner d'un id�al $I$ de $k[X_1,\ldots,X_n]$ et $f \in I$. Alors il existe un unique $r \in k[X_1,\ldots,X_n]$ v�rifiant : \\
\begin{enumerate}
	\item  Tous les termes de $r$ ne sont divisible par aucun des $LT(g_i)$
\item Il existe $g \in I$ tel que $f=g+r$
\end{enumerate}
\end{prop}
\begin{proof}
 L'algorithme de division nous donne l'existence d'un tel $r$. Montrons son unicit�. \\
Supposons, par l'absurde, l'existence de deux restes $r_1$ et $r_2$, $r_1 \neq r_2$ v�rifiant $(1)$ et $(2)$. \\
Alors : $\begin{cases} f=g_1+r_1 \\f=g_2+r_2 \end{cases}$ et donc $r_1-r_2=g_1-g_2 \in I$. \\
D'o�, comme $r_1 \neq r_2$ alors $LT(r_1-r_2) \in \left\langle LT(I) \right\rangle=\left\langle g_1,\ldots,g_s \right\rangle$ et donc $LT(r_1-r_2)$ est divis� par un des $LT(g_i)$ (cf Lemme \ref{}). On obtient donc une contradiction car aucun terme de $r_1$ et $r_2$ n'est divisible par des $LT(g_i)$. D'o� $r_1=r_2$. \\
\end{proof}

\begin{cor} Soit $G=\{g_1,\ldots,g_s\}$ une base de Gr�bner d'un id�al $I$ de $k[X_,\ldots,X_n]$ et $f \in k[X_1,\ldots,X_n]$. Alors $f \in I$ si, et seulement si, le reste de la division de $f$ par $G$ est nul.
\end{cor}

\begin{proof} $\Leftarrow$ : Evident \\
$\Rightarrow$ :  Soit $f \in I$. La d�composition $f=f+0$ respecte les deux conditions de la proposition. Alors par unicit� du reste, le reste de la division de $f$ par $G$ est nul. \\
\end{proof}

\begin{Not} On notera $\overline{f}^F$ le reste de $f$ par le $n$-uple ordonn� $F=\{f_1,\ldots,f_s\}$. Si $F$ est une base de Gr�bner alors on peut consid�rer $F$ comme un ensemble.
\end{Not}

\begin{Def} Soit $f,g \in k[X_1,\ldots,X_n]$ des polyn�mes non nuls.
\begin{enumerate}
\item Si $\multideg(f)=\alpha=(\alpha_1,\ldots,\alpha_n)$ et $\multideg(f)=\beta=(\beta_1,\ldots,\beta_n)$ alors posons $\gamma=(\gamma_,\ldots,\gamma_n)$ o� $\gamma_i=\max(\alpha_i,\beta_i)$. On appelle $X^\gamma$ le plus petit multiple commun de $\LM(f)$ et $\LM(g)$, not� $\PPCM(\LM(f),\LM(g)):=X^\gamma$.
\item Le $S$-polyn�me de $f$ et $g$ est le polyn�me : $S(f,g):=\frac{X^\gamma}{\LT(f)}f-\frac{X^\gamma}{\LT(g)} g$
\end{enumerate}
\end{Def}
\begin{Lemme} Soit $G=\sum_{i=1}^s c_iX^{\alpha_i} g_i$, o� $c_1,\ldots,c_s \in k$ et $\alpha_i+\multideg(g_i)=\delta \in \mathbb{N}^n$ pour $i$ tel que $c_i \neq 0$. \\
Si $LM(G) < X^\delta$ alors il existe des constantes $(c_{jk})$ tel que $G=\sum_{j,k} c_{j,k} X^{\delta-\gamma_{j,k}} S(g_j,g_k)$ o� $X^{\gamma_{j,k}}=\PPCM(\LT(g_j),\LT(g_k))$. De plus, chacun des $X^{\delta-\gamma_{j,k}}$ est strictement inf�rieur � $X^\delta$.
\end{Lemme}
\begin{proof} A faire \end{proof}
\begin{Thm} Soit $I$ un id�al de $k[X_1,\ldots,X_n]$. Alors une base $G=\{g_1,\ldots,g_n\}$ de $I$ est une base de Gr�bner de $I$ si, et seulement si, pour tout couple $(i,j)$, $i \neq j$, $\overline{S(g_i,g_j)}^G=0$
\end{Thm}
\begin{proof} A faire \end{proof}



\section{Algorithme de Buchberger}
\begin{algorithm}

\caption{Algorithme de Buchberger}

\begin{algorithmic}

\REQUIRE $F=(f_1,\ldots,f_s)$

\ENSURE Une base de Gr�bner $G=(g_1,\ldots,g_t)$ de $I $, avec $F \subset G $

\STATE $G:=F $

\REPEAT

\STATE $G':=G $

\FOR{chaque paire $\{p,q\}\in G'^2$, $p \neq q $}

\STATE $S:=\overline{S (p,q)}^{G'} $

\IF{$S \neq 0$}

\STATE $G:=G \cup \{S\} $

\ENDIF
\ENDFOR
\UNTIL $G=G'$

\end{algorithmic}

\end{algorithm}
\begin{Lemme}
Soit $G $ une base de Gr�bner d'un id�al $I $ de $k[X_1,\ldots,X_n] $ et  $P\in G$ tel que $LT (P) \in \left\langle LT(G \setminus \{P\}) \right\rangle$. Alors $G \setminus \{P\} $ est une base de Gr�bner de $I $.
\end{Lemme}
\begin{proof}
Comme $G $ est une base de Gr�bner de $I $ alors $\left\langle LT(G)\right\rangle=\left\langle LT(I) \right\rangle$. Si $LT (P) \in \left\langle LT(G \setminus \{P\}) \right\rangle$ alors $\left\langle LT(G \setminus \{P\} )\right\rangle=\left\langle LT(G )\right\rangle=\left\langle LT(I) \right\rangle$, d'o� $G \setminus \{P\} $ est une base de Gr�bner de $I $.
\end{proof}

\begin{Def}
Une base de Gr�bner minimale $G$ d'un id�al $I $ de $k [X_1,\ldots,X_n] $ est une base de Gr�bner de $I $ telle que :
\begin{enumerate}
	\item  $\forall P \in G,  \LC(P)=1$

 \item $\forall P \in G, \LT(P) \notin \left\langle \LT(G \setminus \{P\}) \right\rangle$
\end{enumerate}
\end{Def}
\begin{Def}

Une base de Gr�bner r�duite $G$ d'un id�al $I $ de $k [X_1,\ldots,X_n] $ est une base de Gr�bner de $I $ telle que :

\begin{enumerate}
	\item $\forall P \in G,  \LC(P)=1$
\item  Pour tout $P \in G$, aucun mon�me de $P $ n'appartient �  $\left\langle \LT(G \setminus \{P\}) \right\rangle$.
\end{enumerate}
\end{Def}
\begin{prop}
Soit $I $ un id�al non nul de $k [X_1,\ldots,X_n] $. Alors, pour un ordre monomial fix�, $I $ a une unique base de Gr�bner r�duite.
\end{prop}
\begin{proof} A faire \end{proof}
\section{Th�or�me d'�limination et d'extension}
\begin{Def}
Soit $I=\left\langle f_1,\ldots,f_s \right\rangle $ un id�al de $k[X_1,\ldots,X_n]$. On appelle $k$�me id�al d'�limination de $I$ l'id�al $I_k$ de $k[X_{k+1},\ldots,X_n]$ d�finit par : $I_k=k[X_{k+1},\ldots,X_n] \cap I$
\end{Def}
 

\begin{Thm}[d'�limination]

Soit $I $ un id�al de $k[X_1,\ldots,X_n]$ et $G$ une base de Gr�bner de $I$ selon l'ordre lexicographique (que l'on notera ici seulement $\geq$). Alors, pour tout $k \in [\![0,n]\!]$, l'ensemble $G_k =G \cap k[X_{k+1},\ldots,X_n]  $ est une base de Gr�bner du $k$�me id�al d'�limination $I_k$.
\end{Thm}

\begin{demo}

Soit $k \in [\![0,n]\!]$. Posons $G=\{g_1,\ldots,g_m\}$ et tel que $G_k=\{g_1,\ldots,g_r\}$ (quitte � renommer les �l�ments). \\
Montrons que $G_k$ est une base de $I_k$. \\
Comme $G_k \subset I_k$ (car $G \subset I$) alors $\left \langle G_k \right\rangle \subset I_k$.\\
Soit $f \in I_k$ alors d'apr�s le th�or�me de division par $G$,il existe $h_1,\ldots,h_m \in k[X_1,\ldots,X_n]$,\\
$f=\sum_{k=1}^m h_ig_i$ car ($G$ est une base de Gr�bner de $I$ et $f \in I$)\\
or pour tout $k>r, g_i > X^{k+1} \geq LM(f) $ et donc aucun terme de $f$ ne peut �tre divisible par un $LT(g_i)$. L'algorithme n'incr�mente pas les $h_k$($k >r$) et donc sont tous nuls.  \\
D'o�, $f=\sum_{k=1}^r h_ig_i$ et donc $f_k \in \left \langle G_k \right\rangle$, ce qui finit de montrer l'�galit� $\left \langle G_k \right\rangle = I_k$.\\
(Le m�me argument permet de montrer que si $f \in I_k$, $\overline{f}^G=\overline{f}^{G_k}$ ).\\
Montrons maintenant que $G$ est une base de Gr�bner de $I_k$.\\
Il suffit, pour cela, de montrer que pour tout $1 \leq i < j \leq r$, $\overline{S(g_i,g_j)}^{G_k}=0$.\\
Soit $i,j \in [\![1,r]\!]$,$ i<j$.\\
Comme $S(g_i,g_j)$ est de la forme $Pg_i+Qg_j$ ($P,Q \in k[X_{k+1},\ldots,X_n]$) et $I_k$ est un id�al alors $S(g_i,g_j) \in I_k \subset I$ d'o� comme $G$ est une base de Gr�bner alors $\overline{S(g_i,g_j)}^{G}=0$ et donc d'apr�s la remarque pr�c�dente, \\
$\overline{S(g_i,g_j)}^{G_k}=0$. Ce qui permet de conclure.
\end{demo}
 

\begin{Thm}[d'extension]

 Soit $I=\left \langle f_1,\ldots,f_s \right\rangle$ un id�al de $\C[X_1,\ldots,X_n]$ et $I_1$ le premier id�al d'�limination.\\
Ecrivons, pour $i \in [\![1,s]\!]$, $f_i$ sous la forme \\
$f_i=g(X_2,\ldots,X_n)X_1^{N_i}$+termes de degr� $<N_i$ en $X_1$\\
o� $N_i \geq 0$ et $g_i \in \mathbb{C}[X_2,\ldots,X_n]$ non nul si $f_i \neq 0$ ($g_i=0$ si $f_i=0$). \\
Supposons qu'on ait une solution partiel $(a_2,\ldots,a_n) \in Z(I_1)$. Si $(a_2,\ldots,a_n) \notin Z(g_1,\ldots,g_s)$ alors il existe $a_1 \in \mathbb{C}$ tel que $(a_1,\ldots,a_n) \in Z(I)$.
\end{Thm}

\begin{cor}

 Soit $I=\left \langle f_1,\ldots,f_s \right\rangle$ un id�al de $\mathbb{C}[X_1,\ldots,X_n]$ et supposons qu'il existe $i \in [\![1,n]\!]$ tel que $f_i$ s'�crit de la forme \\
$f_i=cX_1^{N}$+termes de degr� $<N$ en $X_1$\\
o� $N> 0$ et $c \in \mathbb{C}\neq \{0\}$ non nul. Si $I_1$ est le premier id�al d'�limination de $I$ et $(a_2,\ldots,a_n) \in Z(I_1)$ alors il existe $a_1 \in \mathbb{C}$ tel que $(a_1,\ldots,a_n) \in Z(I)$
\end{cor}

\begin{demo}
Cons�quence imm�diate du th�or�me d'extension.\\
(Comme $g_i=c \neq 0$ alors $Z(g_1,\ldots,g_s) = \emptyset$ et donc $(a_2,\ldots,a_n) \notin Z(g_1,\ldots,g_s)$ pour tout $(a_2,\ldots,a_n) \in \C^{n-1}$).
\end{demo}

\section{G�om�trie}
\subsection{G�n�ralit�s}
\begin{Def}
 Soit $f_1,\ldots,f_s$ des polyn�mes de $k[X_1,\ldots,X_n]$.\\ On appelle vari�t� affine d�finie par $f_1,\ldots,f_s$ l'ensemble : $Z(f_1,\ldots,f_s)=\{(a_1,\ldots,a_n) \in k^n | \forall i \in [\![1,s]\!],f_i(a_1,\ldots,a_n)=0 \}$.
\end{Def}

\begin{Ex} Cercle ;  graphe d'une fonction polynomiale / fonction rationnelle ; Parabolo�de de r�volution ; C�ne ; "Twisted Cubic"
\end{Ex}

\begin{Lemme} : Si $V,W \subset k^n$ sont des vari�t�s affines alors $V \cup W$ et $V \cap W$ aussi.
\end{Lemme}

\begin{proof}
Supposons $V=Z(f_1,\ldots,f_s)$ et $W=Z(g_1,\ldots,g_r)$. \\
Alors $V \cap W=Z(f_1,\ldots,f_s,g_1,\ldots,g_r)$ et $V \cup W=Z(f_ig_j | 1 \leq i \leq s, 1 \leq j \leq r )$ (que l'on notera $Z(f_ig_j)$).\\
Montrons la deuxi�me �galit� : \\
Soit $a=(a_1,\ldots,a_n) \in V$ alors $\forall i \in [\![1,s]\!], f_i(a)=0$ et donc $\forall i \in [\![1,s]\!],\forall j \in [\![1,r]\!],  f_ig_j(a)=0$ d'o� $V \subset Z(f_ig_j)$. On obtient de la m�me fa�on que $W \subset Z(f_ig_j)$. D'o� $V \cup W \subset Z(f_ig_j)$.\\
Soit $a=(a_1,\ldots,a_n) \in Z(f_ig_j)$. Si $a \in V$ alors c'est fini. Sinon, il existe un $i_0 \in [\![1,s]\!]$ tel que $f_{i_0}(a)=0$. Alors,comme pour tout $j \in [\![1,r]\!]$, $f_{i_0}(a)g_j(a)=0$,alors, par int�grit� de $k$, tous les $g_j(a)$ sont nuls et donc $a \in W$.\\
On en d�duit donc $Z(f_ig_j) \subset V \cup W$ et donc l'�galit� voulue.
\end{proof}

\begin{Def} Soit $V=Z(f_1,\ldots,f_s) \subset k^n$. Alors une repr�sentation param�trique de $V$ consiste en des fractions rationnelles   $r_1,\ldots, r_n \in k(X_1,\ldots,X_n)$ telles que les points $(x_1,\ldots,x_n)$ tels que $\forall j \in [\![1,n]\!], x_i=r_i(t_1,\ldots,t_n)$ sont dans $V$.
\end{Def}

\begin{Def} $I$ est dit finement engendr� s'il existe $f_1,\ldots,f_s$ tels que $I=\left\langle f_1,\ldots,f_s \right\rangle$. $\{f_1,\ldots,f_s\}$ est alors appel�e base de $I$.
\end{Def}

\begin{prop} Si $\{f_1,\ldots,f_s\}$ et $\{g_1,\ldots,g_r\}$ sont des bases d'un m�me id�al de $k[X_1,\ldots,X_n]$ alors $Z(f_1,\ldots,f_s)=Z(g_1,\ldots,g_r)$
\end{prop}

\begin{Def} Soit $V \subset k^n$ une vari�t� affine. Alors on pose $I(V):=\{f \in k[X_1,\ldots,X_n] | \forall a\in V, f(a)=0\}$
\end{Def}

\begin{Lemme} Soit $V \subset k^n$ une vari�t� affine. Alors $I(V)$ est un id�al de $k[X_1,\ldots,X_n]$, appel� id�al de $V$.
\end{Lemme}

\begin{proof} $ 0_{k[X_1,\ldots,X_n]} \in I(V)$ car $\forall x \in k^n,0_{k[X_1,\ldots,X_n]}(x)=0$. \\
Soit $f,g \in I(V)$ et $a \in V$ alors $(f+g)(a)=f(a)+g(a)=0$ et donc $f+g \in I(V)$. \\
  Soit $f \in I(V)$, $h \in k[X_1,\ldots, X_n]$ et $a \in V$ alors $(fh)(a)=f(a)h(a)=0h(a)=0$ et donc $fh \in I(V)$. \\
\end{proof}

\begin{Lemme} Soit $f_1,\ldots,f_s \in k[X_1,\ldots,X_n]$. Alors $\left\langle f_1,\ldots,f_s \right\rangle \subset I(Z(f_1,\ldots,f_s))$. L'inclusion r�ciproque n'est pas toujours vraie.
\end{Lemme}

\begin{proof} Soit $f \in \left\langle f_1,\ldots,f_s \right\rangle$ i.e. il existe $h_1,\ldots,h_s$ tels que : $f=\sum_{i=1}^s h_if_i$. Comme $f_1,\ldots,f_s$ s'annule en $V(f_1,\ldots,f_s)$ alors $f=\sum_{i=1}^n h_if_i$ aussi, ce qui permet de dire que $f \in I(Z(f_1,\ldots,f_s))$
\end{proof}

\begin{Ex} $\left \langle X^2,Y^2 \right\rangle \neq I(Z(X^2,Y^2))$.\\
$x^2=y^2=0 \Rightarrow x=y=0$. \\
D'o� $Z(X^2,Y^2)=\{(0,0)\}$ et donc $I(Z(X^2,Y^2))=\left\langle X,Y\right\rangle \neq \left \langle X^2,Y^2 \right\rangle $
\end{Ex}

\begin{prop} Soit $V \subset W$ des vari�t�s affines de $k^n$. Alors :

\begin{enumerate}
	\item  $V \subset W$ ssi $I(V) \supset I(W)$

 \item $V = W$ ssi $I(V) = I(W)$
\end{enumerate}
\end{prop}

\begin{proof} $(1) \Rightarrow (2)$ . Montrons donc $(1)$.\\
$\Rightarrow$ : Supposons $V \subset W$. Soit $f \in I(W)$ alors pour tout $a \in W$ et, en  particulier, pour tout $a \in V$,$f(a)=0$, c'est-�-dire $f \in I(V)$ d'o� $I(W) \subset I(V)$.\\
$\Leftarrow$: Supposons $I(W) \subset I(V)$. Comme $W$ est une vari�t� alors il existe $g_1,\ldots,g_s \in k[X_1,\ldots,X_n]$ tels que $W=Z(g_1,\ldots,g_s)$ alors $g_1,\ldots,g_s \in I(W) \subset I(V)$ et donc les $g_i$ s'annulent sur $V$. \\
Comme $W$ est l'ensemble des points sur lesquels les $g_i$ s'annulent alors $V \subset W$.\\
\end{proof}
\subsection{G�om�trie de l'�limination}
Soit $V=Z(f_1,\ldots,f_s) \subset \C^n $
\begin{Def} 
Soit $\pi_p $ la projection $\C^n \to \C^{n-p} $ d�finie par : $\forall (a_1,\ldots,a_n) \in \mathbb{C}^n, \pi_p (a_1,\ldots,a_n) =(a_{p+1},\ldots,a_n) $.
(Cette application est surjective)
\end{Def}
\begin{Lemme}
Soit $I_p $ le $p $�me id�al d'�limination de l'id�al $\left\langle f_1,\ldots,f_s \right \rangle $ de $\C[X_1,\ldots,X_n] $. Alors, dans $\C^{n-p} $, $\pi_p (V) \subset Z (I_p) $.
\end{Lemme}
\begin{proof}
 Pour montrer cette �galit�, il faut montrer que $\forall a \in \pi_p (V),\forall f \in I_p, f (a)=0$.\\
Soient $a=(a_{p+1},\ldots,a_n) \in \pi_p(V) $ et $f\in I_p  $.\\
Comme $\pi_p $ est surjective alors il existe un $a'=(a_1,\ldots,a_n) $ qui appartient � $V $. Alors $f (a') =0$ (car $f \in \left\langle f_1,\ldots,f_s \right\rangle$). Or comme $f $ ne d�pend que de $X_{p+1},\ldots,X_n$ alors $f (a)=f (a')=0$. \\
\end{proof}
\begin{Thm}  Soit $g_i $ d�fini dans le th�or�me d'extension et $I_1$ le premier id�al d'�limination de $\left\langle f_1,\ldots,f_s \right\rangle$. On a alors l'�galit�, dans $\C^{n-1} $,\\
$Z (I_1)=\pi(V) \cup (Z (g_1,\ldots,g_s) \cap Z (I_1))$
\end{Thm}
\begin{proof}
$\supset $ : c.f. Lemme \ref{} \\
$\subset  $ : Soit $a:=(a_2,\ldots,a_n) \in Z (I_1)$ .
Alors si $a\notin\left\langle g_1,\ldots,g_s \right \rangle $, on a, d'apr�s le th�or�me d'extension, l'existence d'un $a_1 \in \C $ tel que $(a_1,\ldots,a_n) \in V$ et donc  $a \in \pi_1 (V) $.\\
Sinon $a \in \left\langle f_1,\ldots,f_s \right \rangle$ et donc dans $\left\langle f_1,\ldots,f_s \right \rangle \cap V (I_1) $
\end{proof}
\begin{Thm}[de fermeture]

 Soit $V=Z(f_1,\ldots,f_s) \subset \C^n$ et soit $I_p$ le $p$�me id�al d'�limination de $\left\langle f_1,\ldots,f_s \right\rangle$. Alors \ 
\begin{enumerate}
\item $Z(I_p)$ est la plus petite (au sens de l'inclusion) vari�t� contenant $\pi_p(V)$

\item Si $V \neq 0$, alors il existe une vari�t� affine $W \subsetneqq Z(I_p)$ telle que $Z(I_p) \setminus W \subset \pi_p(V)$
\end{enumerate}
\end{Thm}
\begin{cor}

Supposons qu'il existe $i \in [\![1,n]\!]$ tel que $f_i$ s'�crit de la forme : 
$f_i=cX_1^{N}$+termes de degr� $<N$ en $X_1$\\ o� $N> 0$ et $c \in \mathbb{C}\setminus \{0\}$ non nul.\\
Alors $\pi(V)=Z(I_1)$
\end{cor}


\section{Graphe}
\subsection{G�n�ralit�s} 
\begin{Def}

 Un graphe non orient� est un couple $(S,A)$, o� $S$ est un ensemble fini non vide (des �l�ments sont les sommets ) et $A$ est une partie de l'ensemble $\mathcal{P}_2 (S) $ des paires d'�l�ments de $S $ (les �l�ments de $A $ sont les ar�tes). 
\end{Def}  
\begin{Def} 
Soit $G:=(A,S)$ un graphe non orient�. Les sommets $s,t$ sont dits adjacents si $(s,t) \in A$ 
\end{Def}  
\begin{Def} 
Soit $p\in \mathbb{N}^*$.\\ Notons $C_p=\{x_1,\ldots,x_p\}$ un ensemble de couleurs. Un graphe $G:=(A,S)$ est coloriable si on peut associer � chaque sommet de $G$ une couleur de $C_p$ tel que deux sommets adjacents n'aient pas la m�me couleur.  
\end{Def}
  \subsection{Equations polynomiales} 
	Soit $G:=(A,S)$ un graphe non orient� et $p\in \mathbb{N}^*$. \\ 
	Soit $n:=Card(A)$.\\ 
	Associons � chaque sommet de $G$ la variable $x_i$ et � chaque couleur une racine $p$�me de l'unit� \emph{i.e.} $\forall i \in [\![1,n]\!],x_i^p=1$. \\ On impose de plus, que si $x_i$ et $x_j$ sont adjacents alors $x_i \neq x_j$. Cela revient � dire que $\sum_{k=0}^{p-1} x_i^k x_j^{p-1-k}=0$. \\ En effet, $0=x_i^p-x_j^p=\underbrace{(x_i-x_j)}_{\neq 0}\sum_{k=0}^{p-1} x_i^k x_j^{p-1-k}$.  $G$ est coloriable avec $p$ couleurs si, et seulement si, \\ le syst�me $\begin{cases} \forall i \in [\![1,n]\!],x_i^p=1 \\ \forall i,j \in [\![1,n]\!], x_i \text{ et } x_j \text{ sont adjacents }, \sum_{k=0}^{p-1} x_i^k x_j^{p-1-k}=0 \end{cases}$ a une solution 
\section{Ordre}
Soit un ensemble $A$ et une relation d'ordre $\leq$ sur $A$.\\

\begin{Def}
On dit que $\leq $ est un bon ordre si toute partie non vide de $A$ admet un plus petit �l�ment, c'est-�-dire :$\forall  C \subset A,C \neq \emptyset, \exists c \in C, \forall b \in B, c \leq b$.
\end{Def}

\begin{Def}

On dit que $\leq $ est un ordre bien fond� si toute partie non vide de $A$ admet un �l�ment minimal, c'est-�-dire :
$\forall  C \subset A,C \neq \emptyset, \exists c \in C, \forall b \in B, b \leq c \Rightarrow c=b$.
\end{Def}

\begin{prop}

Soit $A$ un ensemble et $\leq $ une relation d'ordre sur $A$.

$\leq$ est total et bien fond� ssi $\leq$ est un bon ordre.
\end{prop}
 \begin{proof}
 Supposons que $\leq$ est total et bien fond�.\\
Soit $C \subset A$ non vide.     \\
Alors il existe un �l�ment minimal $c$ de $C$ (bien fond�)  tel que $ \forall b \in B, b \leq c \Rightarrow c=b$.\\
D'o�  $ \forall b \in B, b > c \text{ ou }  c=b$ car $\leq $ est total. \\
c'est-�-dire $ \forall b \in B, b \geq c$.\\
ou encore que $c$ est le plus petit �l�ment que $C$.\\
 $\leq $ est donc un bon ordre.\\
Supposons que $\leq $ est un bon ordre.\\
Soit $x,y \in A$. Alors $\{x,y\}$ admet un plus petit �l�ment et donc $x \leq y$ ou $y \leq x$.\\
$\leq$ est donc total.\\
Soit $C \subset A$ alors il existe $c \in C$ tel que $\forall b \in C, c\leq b$.\\
Alors si $b \leq c$ alors, par antisym�trie, $b=c$.\\
Cela permet d'en d�duire que $\leq $ est un ordre bien fond�.\\
\end{proof}


\begin{prop}


Soit $A$ un ensemble et $\leq $ une relation d'ordre sur $A$.

$\leq$ est bien fond� ssi il n'existe pas de suite infinie strictement d�croissante.
\end{prop}

\begin{proof}
Montrons cet �nonc� par contrapos�e :\\
$\leq $ n'est pas bien fond�e ssi il existe une suite infinie strictement croissante c'est-�-dire
il existe une partie $S$ de $A$ tel que pour tout $c \in S$, il existe $b \in S$ tel que $c > b$ \\
(car $non(A \Rightarrow B) \Leftrightarrow (A \text{ et } non(B))$ et donc $ (b \leq c \Rightarrow c=b) \Leftrightarrow (b \leq c \text{ et } b \neq c)  \Leftrightarrow (b < c)$\\
Soit $\alpha_1 \in S$ alors il existe $\alpha_2 \in S$  tel que $\alpha_1>\alpha_2$.\\
En it�rant ce processus, on construit une suite $(\alpha_i)_{i \in \mathbb{N}}$ strictement d�croissante.\\
R�ciproquement, supposons l'existence d'une telle suite alors l'ensemble $\{a_i | i \in \mathbb{N}\}\subset A$ n'admet pas d'�l�ment minimal donc $\leq$ n'est pas bien fond�.
\end{proof}





\section{Anneau noeth�rien}
\begin{Defprop}

Soit $A$ un anneau commutatif. Alors les deux conditions suivantes sont �quivalentes :

\begin{enumerate}
	\item  Toute suite croissante d'id�aux de $A$ est stationnaire.
\item Tout id�al $I$ de $A$ est de type fini c'est-�-dire qu'il existe une famille finie $f_1,\ldots,f_n \in I$ telle que : $I=\left\langle f_1,\ldots,f_n  \right\rangle$
\end{enumerate}
Un tel anneau est alors dit noeth�rien.
\end{Defprop}

\begin{proof}
 Montrons $(1) \Rightarrow (2)$.\\
Supposons donc que toute suite croissante d'id�aux de $A$ est stationnaire.\\
Soit $\mathscr{I}$ un id�al de $A$ et consid�rons la suite d'id�al $(I_n)$ d�finie par : $I_0=\left\langle 0  \right\rangle$ et pour tout $n \in \mathbb{N}$,$I_{n+1}=\left\langle I_n,a_{n+1}  \right\rangle$ o� $a_{n+1} \in \mathscr{I} \setminus I_n$ si $I_n \neq \mathscr{I}$ et $I_{n+1}=I_n$ sinon.\\
Alors $(I_n)$ est croissante et plus pr�cis�ment, elle est strictement croissante tant que $I_n \neq \mathscr{I}$ et constante sinon. \\
On en d�duit que $(I_n)$ est stationnaire (c.f. (1)) et donc qu'il existe $N \in \mathbb{N}$ tel que :\\
$\forall n \geq N,\mathscr{I}=I_n=I_N=\left\langle a_1,\ldots,a_n  \right\rangle$ .\\
Montrons maintenant que $(2) \Rightarrow (1)$. \\
 Supposons donc que tout id�al $I$ de $A$ est de type fini.\\
Soient  $(I_n)$ une suite croissante d'id�aux et $I:=\bigcup_{n \in \mathbb{N}} I_n$. \\
 Par hypoth�se, il existe donc $a_1,\ldots,a_p \in I$ tel que $I=\left\langle a_1,\ldots,a_p  \right\rangle$. De plus, comme $a_1,\ldots,a_p \in \bigcup_{n \in \mathbb{N}} I_n$ alors pour tout $a_i$ il existe $n_i$ tel que $a_i \in I_{n_i}$ avec $1 \leq i \leq p$. \\
Posons maintenant $N:= max_{1 \leq i \leq p} n_i$.\\
Alors pour tout $n \geq N$, $a_1,\ldots,a_p \in I_n$. D'o� :\\
$\left\langle a_1,\ldots,a_p \right\rangle \subset I_N \subset I_n \subset I=\left\langle a_1,\ldots,a_p \right\rangle$.\\
 Et donc pour tout $n \geq N$,$I=I_n=I_N$. $(I_n)$ est donc stationnaire.\\
\end{proof}

\begin{Ex} Tout anneau principal est noeth�rien car chaque id�al d'un anneau principal $A$  est de la forme $aA$ o� $a \in A$.\\
En particulier, tout corps est noeth�rien (les id�aux d'un corps sont $\{0\}$ et lui-m�me).
\end{Ex}

\begin{Thm}[de la base de Hilbert]

Soit $A$ un anneau noeth�rien. Alors $A[X]$ est aussi un anneau noeth�rien.
\end{Thm}

\begin{proof}

Soient $I$ un id�al de $A[X]$, $J$ l'id�al engendr� par les coeff et pour tout $n \in \mathbb{N}$, $J_n=\left\langle \{a | aX^n+\sum_{k=0}^{n-1} a_kX^k \in I\} \right\rangle$ des id�aux de $A$.\\
Comme $A$ est noeth�rien alors il existe $x_1,\ldots,x_r \in I$,tels que $J=\left\langle x_1,\ldots,x_r\right\rangle$ et pour tout $n \in \mathbb{N}, y_{1,n},\ldots,y_{m_n,n}$ tels que $J_n=\left\langle y_{1,n},\ldots,y_{m_n,n} \right\rangle$.\\
Il existe donc des polyn�mes $Q_1,\ldots,Q_r$ de $I$ ayant pour coefficient dominant $x_i$ et pour tout $n \in \mathbb{N}$, des polyn�mes $R_{1,n},\ldots R_{m_n,n}$ qui ont pour coefficient en $X^n$ �gale � $y_{m_n,n}$.\\
Montrons que $I=\left\langle  Q_1,\ldots,Q_r, R_{1,1},\ldots,R_{m_1,1},\ldots,R_{1,N},\ldots,R_{m_N,N} \right \rangle$ o� $N:=max_i \; deg(Q_i)$.\\
Notons $I'$  cet id�al (inclus dans $I$ car engendr� par des �l�ments de $I$) et montrons,par r�currence sur le degr� de $P$, que si $P:=\sum a_iX^i \in I'$ alors $P \in I$.\\

\textbf{Initialisation }: Si $P=0$ alors $P \in I$ et $P \in I'$ (car ce sont des sous-groupes additifs de $A[X]$)\\

\textbf{H�r�dit�} : Soit $d \in \mathbb{N}$ et supposons que pour tout polyn�mes de degr� $P$ de degr� strictement inf�rieur � $d$ que si $P \in I$ alors $P \in I'$.\\
Soit $P:=\sum_{k=0}^d a_k X^k \in I$.
\begin{itemize}
\item Si $d \leq N-1$ alors $a_d \in J_d$, il existe donc $\lambda_1,\ldots,\lambda_{m_d}$ tel que $a_d=\sum_{k=1}^{m_d} \lambda_k y_{k,d}$. On en d�duit que $T:=P-\sum_{k=1}^{m_d} \lambda_k R_{k,d}$ est de degr� inf�rieur � $n-1$. Comme $P$ et les $R_{k,d}$ sont dans $I$ alors $T$ aussi et par hypoth�se de r�currence $T \in I$. Comme les $R_{k,d}$ sont aussi dans $I'$ alors $P=T+\sum_{k=1}^{m_d} \lambda_k R_{k,d}$ est dans $I'$.
\item Si $d \geq N$, alors $a_d \in J$,il existe donc $\lambda_1,\lambda_r \in A$ tel que $a=\sum_{k=1}^r \lambda_k x_k$ et donc $P-\sum_{k=1}^m \lambda_i X^{n-deg(Q_i)} Q_i$ est de degr� inf�rieur � $n-1$. On en d�duit comme pour le cas $d \leq N-1$ que $P\in I'$.
\end{itemize}
Conclusion : D'apr�s le principe de r�currence, $I \subset I'$  et donc $I=I'$
$A[X]$ est donc noeth�rien.
\end{proof}
\begin{cor}
$\Z[X_1,\ldots,Z_n]$,$k[X_1,\ldots,X_n]$ sont des anneaux noeth�riens.
\end{cor}





\section{Alg�bre}
\subsection{Polyn\^omes irr\'eductibles et factorisation}
 \begin{Def}
 Un polyn\^ome $P \in k[X_1,\ldots,X_n]$ est irr\'eductible sur $k$ si $P$ est non constant et qu'il n'est pas le produit de deux polyn\^omes non constants de $k[X_1,\ldots,X_n]$
 \end{Def} 
\begin{prop}
 Tout polyn\^ome non constant de $k[X_1,\ldots,X_n]$ peut s'\'ecrire comme produit de polyn\^omes irr\'eductible sur $k$. 
\end{prop}
 \begin{Thm} 
Soit $P \in k[X_1,\ldots,X_n]$ irr\'eductible sur $k$ et supposons que $P$ divise le produit $QR$, avec $Q,R \in k[X_1,\ldots,X_n]$. Alors $P$ divise $Q$ ou $R$. 
\end{Thm} 
\begin{Thm} 
 Tout polyn\^ome non constant $f\in k[X_1,\ldots,X_n]$ peut s'\'ecrire comme un produit $f=f_1 \ldots f_r$ d'irr\'eductible sur $k$. De plus, $f=g_1\ldots g_s$ est une autre factorisation en irr\'eductible sur $k$, alors $r=s$ et les $g_i$ peuvent être permut\'es de tel sorte que pour tout $i$,  $g_i$ soit un multiple de $f_i$ . 
\end{Thm}
\section{R�sultant}
Soient $R$ un anneau commutatif int\`egre de corps de fractions $L$ ainsi que : \\
$A:=\sum_{k=0}^p a_kX^k \in R_p[X]$ et $B:=\sum_{k=0}^q b_kX^k \in R_p[X]$.\\
On appelle la matrice de Sylvester la matrice : \\
$S_{p,q}(A,B)=\begin{pmatrix}
a_0 & 0 &\cdots & 0 & b_0 & 0 &0 & \cdots & 0 & 0 \\
a_1 & a_0& \ddots & 0 &b_1 & b_0 &0 & \ddots & 0 & 0 \\
\vdots & a_1 &\ddots & 0 &\vdots & b_1 &b_0 & \ddots & 0 & 0 \\
\vdots & \vdots &\ddots & a_0 &\vdots & \vdots & b_1 & \ddots & 0 & 0 \\
a_ {p-1} & \vdots& \ddots & a_1 &b_q & \vdots &\vdots & \ddots & b_0 & 0 \\
a_p & a_{p-1} &\ddots & \vdots &0 & b_q &\vdots & \ddots & b_1 & b_0 \\
0 & a_p &\ddots & \vdots &0 & 0 &b_q & \ddots & \vdots & b_1 \\
\vdots & \vdots &\ddots & \vdots &\vdots & \vdots &0 & \ddots & \vdots & \vdots \\
\vdots & \vdots &\ddots & a_{p-1} &\vdots & \vdots &0 & \ddots & b_q & \vdots \\
0 & 0& \ldots & a_p &0 & 0 &0 & \cdots & 0 & b_q \\
\end{pmatrix}$. \\
On notera par $Res_{p,q}(A,B)$ le d\'eterminant de $S_{p,q}(A,B)$
\begin{prop}
$Res_{p,q}(A,B)$ est nul si, et seulement si, il existe $P \in R_{q-1}[X]$ et $Q \in R_{p-1}[X]$ non tous deux nuls tels que $AP+BQ=0$
\end{prop}
\begin{prop}
Il existe $P \in R_{q-1}[X]$ et $Q \in R_{p-1}[X]$ non tous deux nuls tels que $AP+BQ=Res_{p,q}(A,B)$
\end{prop}
\begin{Def}
Le r\'esultant des polyn\^omes $A,B \in R[X]$ de degr\'es respectifs $p,q \geq 0$ est l'\'el\'ement $Res(A,B)=Res_{p,q}(A,B)$
\end{Def}
\begin{Rq}
Lien entre les valeurs de $Res_{p,q}(A,B)$ et de $Res(A,B)$ : 
\begin{itemize}
    \item Si $p=deg(A)$ et $q=deg(B)$ alors, par d\'efinition, $Res(A,B)=Res_{p,q}(A,B)$
    \item Si $p=deg(A)$ et $q>deg(B)$ alors $Res_{p,q}(A,B)=((-1)^p a_p)^{q-deg B}Res(A,B)$
    \item Si $p>deg(A)$ et $q=deg(B)$ alors $Res_{p,q}(A,B)=b_q^{p-deg A}Res(A,B)$
    \item Si $p>deg(A)$ et $q=deg(B)$ alors $Res_{p,q}(A,B)=0$
\end{itemize}
\end{Rq}
\begin{prop}
Soit $A=QB+A_1$ une division euclidienne, avec $A_1 \neq 0$. Alors, avec les m\^emes notations que pr\'ec\'edemment, $Res(A,B)=b_q^{deg(A)-deg(A_1} Res(A_1,B)$
\end{prop}
\begin{Lemme}
Si $B=(X-\beta)*C$, alors $Res(A,B)=A(\beta)Res(A,C)$
\end{Lemme}
\begin{Thm}
Si $A:=a(X-\alpha_1)\ldots (X-\alpha_p)$ et $B:=b(X-\beta_1)\ldots(X-\beta_q)$, alors :
$Res(A,B)=b^pA(\beta_1)\ldots A(\beta_q)$
$=b^pa^q \prod_{i=1}^p \prod_{j=1}^q (\beta_j-\alpha_i)$
$=(-1)^{pq} a^q B(\alpha_1) \ldots B(\alpha_q)$
\end{Thm}
\begin{cor}
Supposons le corps $L$ alg\'ebriquement clos. Alors $Res(A,B)=0$ si, et seulement si, les polyn\^omes $A$ et $B$ ont une racine commune.
\end{cor}


\section{Implicitation}
Soit $S$ l'ensemble param\'etr\'e par le syst\`eme suivant : $ \begin{cases}
x_1=f_1(t_1,\ldots,t_m)& \\
\vdots & (\dagger) \\
x_n=f_n(t_1,\ldots,t_m)& \\
\end{cases} $ o\`u $f_i \in k[T_1,\ldots,T_m]$ et $(t_1,\ldots,t_m) \in k^m$ .\\
On peut voir $S$ comme l'image de la fonction $F : k^m \to k^n$ d\'efinie par : \\
$\forall t \in k^m, F(t)=(f_1(t),\ldots,f_n(t))$.\\
$S$ n'est pas n\'ecessairement une vari\'et\'e affine (cf exercices).\\
Le syst\`eme $(\dagger)$ d\'efini tout de même une vari\'et\'e $V=Z(X_1-f_1,\ldots,X_n-f_n) \subset k^{n+m}$.\\
On a donc $V=\{(t_1,\ldots,t_n,x_1,\ldots,x_m)\in k^{n+m} | \forall i \in [\![1,m]\!],x_i-f_i(t_1,\ldots,t_n)=0 \}$\\
D'o\`u,  $V=\{(t_1,\ldots,t_n,f_1(t_1,\ldots,t_n),\ldots,f_m(t_1,\ldots,t_n))\in k^{n+m} | (t_1,\ldots,t_m) \in k^m \} (*)$. Autrement dit, $V$ est le graphe de $F$.\\
Soient $\begin{array}{ccccc}
i & : & k^m & \to & k ^{n+m} \\
 & & (t_1,\ldots,t_m) & \mapsto & (t_1,\ldots,t_n,f_1(t_1,\ldots,t_n),\ldots,f_m(t_1,\ldots,t_n))\\
\end{array}$\\
 et $\begin{array}{ccccc}
\pi_m & : & k^{n+m} & \to & k^n \\
 & & (t_1,\ldots,t_n,x_1,\ldots,x_m) & \mapsto & (t_1,\ldots,t_n) \\
\end{array}$
Alors, on a : \\
 \xymatrix{
   &k^{n+m} \ar[rd]^{\pi_m}&  \\
    k^m \ar[ru]^i\ar[rr]^F&& k^n
  }\\
i.e. $F=\pi_m \circ i$.\\
Avec $(*)$, on a $i(k^m)=V$ et donc $\pi_m(V)=F(k^m)$.\\
Autrement dit, l'image d'une param\'etrisation est la projection de son graphe.
\begin{Thm}
Soit $F:\C^m \to \C^n$ une fonction d\'etermin\'ee par la param\'etrisation polynomiale $(\dagger)$.\\ Soit $I$ l'id\'eal $\left\langle X_1-f_1,\ldots,X_n-f_n \right\rangle \subset \C[T_1,\ldots,T_m,X_1,\ldots,X_n]$ et $I_m$ son $m$ \`eme id\'eal d'\'elimination. Alors $Z(I_m)$ est le plus petit id\'eal de $\C^n$ contenant $F(\C^m)$
\end{Thm}
\section{Nullstellensatz}
\input{Parties/nullstellensatz}
\nocite{*}
\bibliographystyle{plain}
\bibliography{biblio}
\end{document}
